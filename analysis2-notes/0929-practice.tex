\section*{September 29th, 2022 (Practice)}

삼각함수와 지수함수를 엄밀하게 construct 하는 방법.

\defn. Define the \textbf{exponential function} as
\[
    E(z) = \sum_{n=0}^\infty \frac{z^n}{n!}, \quad (z \in \C).
\]

Check that
\[
    E(z)E(w) = E(z + w), \quad (z, w, \in \C).
\]
Be careful when you switch infinite summations. We directly get
\[
    E(z)E(-z) = E(0) = 1, \quad (z \in \C).
\]
This shows that \(E(z) \neq 0\), \(E(x) > 0\) even if \(x < 0\). Also,
\begin{center}
    \(E(x) \ra \infty\) as \(x \ra \infty\) \quad and \quad \(E(x) \ra 0\) as \(x \ra -\infty\).
\end{center}
Also check that \(E'(x) = E(x) > 0\).

\defn. \note{Constant \(e\)} Define
\[
    e = E(1) = \sum_{n=0}^\infty \frac{1}{n!} = 1 + \frac{1}{1!} + \frac{1}{2!} + \cdots
\]

Then for \(n \in \N\),
\[
    E(n) = E(\overbrace{1 + 1 + \cdots + 1}^{n\text{ times}}) = (E(1))^n = e^n.
\]
Similar process can be done for \(n \in \Z\). For \(1/m \in \Q\),
\[
    E\left(\frac{1}{m}\right)^m = E\overbrace{\left(\frac{1}{m} + \frac{1}{m} + \cdots + \frac{1}{m}\right)}^{m\text{ times}} = E(1) = e,
\]
thus \(E\left(\frac{1}{m}\right) = \sqrt[m]{e}\). For \(n/m \in \Q\),
\[
    E\left(\frac{n}{m}\right) = E\overbrace{\left(\frac{1}{m} + \frac{1}{m} + \cdots + \frac{1}{m}\right)}^{n\text{ times}} = E\left(\frac{1}{m}\right)^n = \left(\!\sqrt[m]{e}\right)^n = e^{n/m}.
\]
For \(r \in \R\),
\[
    e^r = \sup\{E(q) : q \in \Q, q < r\} = \inf\{E(q) : q \in \Q, q > r\} = \lim_{q \ra r} E(q).
\]
Using the monotonicity and continuity of \(E(z)\), gives
\[
    E(x) = e^x, \quad (x \in \R).
\]

\thm{8.6} For every \(n \in \N\),
\[
    \lim_{x \ra\infty} \frac{x^n}{e^x} = 0.
\]

\pf
\[
    e^x > \frac{x^{n+1}}{(n+1)!} \implies \frac{x^n}{e^x} < \frac{(n+1)!}{x}.
\]
Now take the limit \(x \ra \infty\).

The exponential function is strictly increasing and bijective, so it has an inverse function \hspace*{-1px}\(L\).

\defn. Define the \textbf{logarithmic} function \(L\) as
\begin{center}
    \(L(E(x)) = x\) for \(x \in \R\) \quad or \quad \(E(L(y)) = y\) for \(y > 0\).
\end{center}
We write
\[
    L(y) = \log y, \quad (y > 0).
\]

Using the chain rule gives
\[
    L'(E(x)) E'(x) = 1 \implies L'(y) = \frac{1}{y},
\]
and
\[
    L(x) = \int_1^x \frac{1}{t}\d{t}.
\]

Check that
\[
    L(xy) = L(x) + L(y), \quad (x, y > 0)
\]
\[
    E\left(\frac{1}{m}L(x)\right) = x^{1/m}, \quad E\left(\frac{n}{m}L(x)\right) = x^{n/m} \qquad (x > 0, n, m \in \N)
\]
Therefore for \(\alpha \in \Q\),
\[
    x^\alpha = E(\alpha L(x)) = e^{\alpha \log x}, \quad (x > 0),
\]
and differentiating gives
\[
    (x^\alpha)' = E(\alpha L(x))\cdot \frac{\alpha}{x} = \alpha x^{\alpha - 1}.
\]

\thm. For every \(\alpha > 0\),
\[
    \lim_{x \ra\infty} \frac{\log x}{x^\alpha} = 0.
\]

\pf Take \(0 < \epsilon < \alpha\) and \(x > 1\). Then
\[
    \frac{\log x}{x^\alpha} = \frac{1}{x^\alpha} \int_1^x \frac{1}{t}\d{t} < \frac{1}{x^\alpha} \int_1^x \frac{t^\epsilon}{t} \d{t} = \frac{1}{x^\alpha} \frac{x^\epsilon - 1}{\epsilon} < \frac{1}{\epsilon x^{\alpha - \epsilon}}.
\]
Now take the limit \(x \ra \infty\).
We also have the series representation
\[
    \log(1+x) = \sum_{n=1}^\infty \frac{(-1)^{n-1}}{n}x^n = x - \frac{x^2}{2} + \frac{x^3}{3} + \cdots.
\]

\medskip

\subsubsection*{Trigonometric Functions}

\defn. Define
\[
    C(x) = \frac{E(ix) + E(-ix)}{2}, \quad S(x) = \frac{E(ix) - E(-ix)}{2i}.
\]
From the series representation of \(E(x)\), we see that
\[
    C(x) = \sum_{n=0}^\infty \frac{(-1)^n}{(2n)!}x^{2n}, \quad S(x) = \sum_{n=0}^\infty \frac{(-1)^n}{(2n+1)!} x^{2n+1}.
\]

Also,
\[
    E(ix) = C(x) + iS(x), \quad (x\in \R).
\]
and
\[
    \abs{E(ix)}^2 = E(ix)\overline{E(ix)} = E(ix) E(-ix) = E(0) = 1
\]
So we know that
\[
    C^2(x) + S^2(x) = 1
\]
and from the power series representation,
\[
    S'(x) = C(x), \quad C'(x) = -S(x).
\]
There exists positive numbers \(x\) such that \(C(x) = 0\). (Proof in text) Let \(x_0\) be the smallest positive number such that \(C(x_0) = 0\).\footnote{\(C\inv(0) \cap \R^+\) 는 닫힌 집합. 아래로 유계이고 닫혀 있으므로, \(\inf\) 가 존재.}

\defn. Define the number \(\pi\) by \(\pi = 2x_0\).

We know that \(C(x) > 0\) for \(x \in [0, \pi/2]\), so \(S(\pi/2) = 1\) (\(S\) is increasing on \((0, \pi/2)\)). Thus
\[
    e^{2\pi i} = e^{i\frac{\pi}{2} \cdot 4} = i^4 = 1.
\]

We want to show that \(E\) hs period \(2\pi i\). If there exists \(T \in (0, 2\pi)\) such that \(e^{xi} = e^{(x+T)i}\), \(e^{Ti} = 1\). Then \(e^{\frac{T}{4}i}\) is one of \(\pm 1, \pm i\). Since \(0 < T < 2\pi\), \(0 < T/4 < \pi/2\). But \(0 < C(\pi/4) < 1\), while
\[
    \Re(e^{\frac{\pi}{4}i}) = 1 \text{ or } 0,
\]
which leads to a contradiction.

We can prove the trig identities by
\[
    C(x+y) + i S(x+y) = e^{i(x+y)} = e^{ix}e^{iy} = (C(x) + iS(x))(C(y) + iS(y))
\]
and expanding the last expression.

Note that \(e^{ix}\) defined on \([0, 2\pi)\) is an injective function, and \(\abs{e^{ix}} = 1\). Consider the curve \(e^{ix}\) for \(0 \leq x \leq \theta\). Then the length of this curve is
\[
    \int_0^\theta \abs{\frac{d}{dx}e^{ix}}\d{x} = \int_0^\theta \d{x} = \theta.
\]
This gives the definition of radian angles. On the unit circle with a point \((x, y)\),
\[
    \cos \theta = x = \Re(e^{ix}) = C(x),\quad \sin\theta = y = \Im(e^{ix}) = S(x).
\]

\pagebreak
