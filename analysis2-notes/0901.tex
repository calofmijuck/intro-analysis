\section*{Introduction \& Notice}

\begin{itemize}
    \item 7, 8장 나가고 중간고사, 11장 나가고 기말고사
    \item 연습 시간이 있는 수업 (목 6:30 \(\sim\) 8:20)\footnote{가능하면 1시간 반 안에 끝내라고 하심 ㅋㅋ}
    \item 오늘 연습 시간: 지난학기 배운 내용 중 필요한 내용 복습
\end{itemize}

\chapter{Sequences and Series of Functions}

\section*{September 1st, 2022}

함수들을 각각 보는 것보다 \textbf{함수들의 공간을 이해}하는 것이 해석학의 핵심이다! 공간을 이해해야 미분방정식도 풀고 실제 현실의 문제들을 풀 수 있는 것입니다.

\(\R^n\)을 단순히 좌표들의 모임으로 보는 것이 아니라, 거리 구조를 주고 열린/닫힌 집합과 같은 위상 구조를 줬었습니다. 이 전에 했던게 \textbf{수열의 수렴과 발산, 코시 수열}이죠. 이런 것들을 바탕으로 공간의 위상적 성질을 조금 더 효율적으로 공부할 수 있었습니다.

즉, 어떤 공간을 배우기 위해서는 수열의 수렴이나 발산을 배워야 합니다. \textbf{따라서 우리는 함수들의 공간을 공부하기 위해 우선 함수열을 공부합니다.}

\medskip

Suppose \(E\) is a set\footnote{사실은 \textit{metric} space 이다.}, and let \(f_n: E \ra \C\) for all \(n \in \N\). Then
\[
    \seq{f_n}_{n=1}^\infty
\]
is a sequence of (complex-valued) function.

\medskip

수열을 공부했으니 수열의 \textbf{수렴}을 정의해야 할 것입니다.

\defn{7.1} \note{Pointwise Convergence} \(\seq{f_n}_{n = 1}^\infty\) converges \textbf{pointwise} on \(E\), if for each \(x \in E\) the sequence \(\seq{f_n(x)}_{n=1}^\infty\) converges in \(\C\).

In other words, for each \(x \in E\), there exists \(a_x \in \C\) and
\begin{center}
    \(\forall \epsilon > 0, \exists N_x \in \N\) such that \(n \geq N_x \implies \abs{f_n(x) - a_x} < \epsilon\).
\end{center}

\defn. If \(\seq{f_n}\) converges pointwise, we can define a function \(f\) by
\[
    f(x) = \lim_{n \ra \infty} f_n(x) \quad (x\in E)
\]

We say that
\begin{itemize}
    \item \(f\) is the \textit{limit} or \textit{limit function} of \(f_n\).
    \item \(\seq{f_n}\) to \(f\) pointwise on \(E\).
\end{itemize}

\defn. If \(\sum f_n(x)\) converges (pointwise) for every \(x \in E\), we can define
\[
    f(x) = \sum_{n=1}^\infty f_n(x) \quad (x \in E)
\]
and the function \(f\) is called the \textit{sum} of the series \(\sum f_n\).

\recall \(f: (E, d) \ra \C\) is continuous on \(E\) \miff \(f\) is continuous at all \(x \in E\).

\recall \note{Theorem 4.6} If \(p \in E\) and \(p\) is a limit point of \(E\),
\begin{center}
    \(f\) is continuous at \(p\) \miff \(\ds \lim_{x \ra p} f(x) = f(p)\)
\end{center}

\question Suppose \(\seq{f_n}\) is a sequence of functions. Does the limit function or the sum of the series preserve important properties?
\begin{enumerate}
    \item If \(f_n\) is continuous, is \(f\) continuous?
    \item If \(f_n\) is differentiable/integrable, is \(f\) differentiable/integrable?
\end{enumerate}

For (1), the question is equivalent to the following:

\textit{If \(p\) is a limit point, does the following hold?}
\[
    \lim_{x\ra p} \lim_{n\ra \infty} f_n(x) \overset{?}{=} \lim_{n\ra \infty} \lim_{x\ra p}f_n(x)
\]

And the answer is \textbf{No}.

실수열이 주어졌을 때, 극한을 계산하여 극한값이 실수라는 것은 굉장히 중요한 것입니다. 우리가 \textbf{연속}함수열의 수렴을 정의할 때, 극한이 되는 함수 또한 \textbf{연속}이 되기를 기대하는 것은 굉장히 자연스러운 일입니다. 하지만 점별수렴하는 연속함수열의 극한은 연속이 아닐 수 있습니다. 즉, 점별수렴은 연속함수공간에서의 `수렴'으로 정의하기에는 부족합니다.

\ex{7.2} Suppose \(\ds a_{m, n} = \frac{m}{m + n}\) for \(m, n\in \N\). We see that
\[
    \lim_{n \ra \infty} \lim_{m \ra \infty} a_{m, n} = 1 \neq 0 = \lim_{m \ra \infty} \lim_{n \ra \infty} a_{m, n}
\]

\ex. Define
\[
    f_n(x) = \begin{cases}
        0       & \paren{\frac{1}{n} \leq x \leq 1} \\
        -nx + 1 & \paren{0 \leq x < \frac{1}{n}}
    \end{cases}
\]
then we can easily see that
\[
    f(x) = \lim_{n\ra \infty} f_n(x) = \begin{cases}
        0 & (0 < x \leq 1) \\
        1 & (x = 0)
    \end{cases}
\]
Thus \(f\) is not continuous at \(x = 0\).

\ex. Define \(f_n: \R \ra \R\) as
\[
    f_n(x) = \frac{x^2}{(1+x^2)^n} \quad (n = 0, 1, 2, \dots)
\]
by direct calculation,
\[
    f(x) = \sum_{n = 0}^\infty f_n(x) = \sum_{n = 0}^\infty \frac{x^2}{(1+x^2)^n} = 1 + x^2 \quad (x \neq 0)
\]
since this is a geometric series when \(x \neq 0\). If \(x = 0\), \(f(x) = 0\) and \(f\) is not continuous.

\question Does the limit function preserve Riemann integrability?

\ex. For \(m = 1, 2, \dots\), define
\[
    f_m(x) = \lim_{n\ra \infty} \left(\cos m! \pi x\right)^{2n} = \begin{cases}
        1 & (m!x \in \Z)    \\
        0 & (m!x \notin \Z)
    \end{cases}
\]
We see that \(f_m(x)\) is Riemann integrable. However,

\quad \claim.
\[
    f(x) = \lim_{m\ra \infty} f_m(x) = \begin{cases}
        1 & (x\in \Q)         \\
        0 & (x \in \R \bs \Q)
    \end{cases}
\]
and \(f(x)\) is nowhere continuous, also not Riemann integrable.

\quad \pf Suppose \(x = p/q \in \Q\). (\(p, q\in \Z\)) If we take \(m \geq q\), we see that \(m! x \in \Z\). Thus \(f_m(x) = 1\). If \(x \notin \Q\), \(m! x\) can never be in \(\Z\) and \(f_m(x) = 0\).

\medskip

Uniform continuity를 할 때 uniform이 어디서 나오죠? 해석학에서 그 점에서 뭐가 성립한다, 그러면 그 점과 그 근방에서만 확인하면 됐었죠. Continuity는 local property죠. 그런데 uniform continuity는 전체가 다 uniform하게 성립한다는 의미입니다.

\recall \(f: (X, d) \ra (Y, d)\) is \textbf{uniformly continuous} on \(X\)\footnote{Subspace of metric space is also a metric space.} if
\begin{center}
    \(\forall \epsilon > 0, \exists \delta > 0\) such that \(d_X(p, q) < \delta \implies d_Y(f(p), f(q)) < \epsilon\)
\end{center}
즉, 모든 점에서 똑같이 잡을 수 있다!

\recall \note{Theorem 4.19, Heine-Cantor} If \(X\) is compact and \(f\) is continuous on \(X\), then \(f\) is uniformly continuous on \(X\).\footnote{갑자기 왜 uniform continuity 얘기를 하냐, 헷갈리지 말고 기억하시라고!}

이제부터 나오는 uniform convergence는 sequence에 관한 것입니다!

\defn{7.7} \note{Uniform Convergence} Suppose \(f_n: E \ra \C\) is a sequence of functions. \(\seq{f_n}_{n=1}^\infty\) \textbf{converges uniformly} on \(E\) to a function \(f\) if
\begin{center}
    \(\forall \epsilon > 0, \exists N\in\N\) such that \(\forall x \in E, n\geq N \implies \abs{f_n(x) - f(x)} \leq \epsilon\).
\end{center}
Also, we say that the series \(\sum f_n(x)\) converges uniformly on \(E\) if the sequence of partial sums \(\seq{\sum_{k=1}^n f_k(x)}\) converges uniformly on \(E\).

점별수렴의 경우 \(N_x \in \N\) 이 \(x \in E\) 에 의존하지만, 고른수렴의 경우 \(N\) 이 \(x\)와 무관합니다!

[똑같은 \(\epsilon\)-띠를 둘러서 \(y = f(x)\) 의 근방 안에 \(f_n(x)\) (\(n \geq N\)) 가 모두 들어가 있어야 한다]는 의미에서 \textit{uniform}입니다. 한꺼번에 \(\epsilon\)으로 누를 수 있다는 것입니다.

\thm{7.9} Suppose
\[
    f(x) = \lim_{n \ra \infty} f_n(x) \quad (x \in E)
\]
Then \(f_n \ra f\) converges uniformly on \(E\) if and only if
\[
    \lim_{n \ra \infty} \sup_{x \in E} \abs{f_n(x) - f(x)} = 0
\]
which can also be written as
\begin{center}
    \(\forall \epsilon > 0, \exists N\in \N\) such that \(\ds n \geq N \implies \sup_{x \in E} \abs{f_n(x) - f(x)} \leq \epsilon\)
\end{center}

\notation \(f_n \ra f\) uniformly on \(E\) \miff \(f_n \uc f\) on \(E\).\footnote{교수님: 책에서는 나중에 \(\norm{f_n(x) - f(x)}_\infty \ra 0\) 으로 적었던 것 같은데...}

\thm{7.8} \note{Cauchy Criterion for Uniform Convergence}
\(f_n \uc f\) on \(E\) \miff
\begin{center}
    \(\forall \epsilon > 0, \exists N \in \N\) such that \(n, m \geq N \implies \ds \sup_{x \in E} \abs{f_n(x) - f_m(x)} \leq \epsilon\).\footnote{Uniform Cauchy Property. 실수에서 알고있던 성질과 동일합니다.}
\end{center}

\pf\\
(\mimp) For given \(\epsilon > 0\), fix \(x \in E\). Since \(f_n\) converges uniformly on \(E\), we can find \(N \in \N\) such that for \(n, m \geq N\),
\[
    \abs{f_n(x) - f_m(x)} \leq \abs{f_n(x) - f(x)} + \abs{f(x) - f_m(x)} \leq \frac{\epsilon}{2} + \frac{\epsilon}{2} = \epsilon
\]
(\(\impliedby\)) Uniform Cauchy property implies that \(\seq{f_n}\) is a Cauchy sequence in \(\C\). By the completeness of \(\C\), the limit function \(f(x)\) exists. Now we show that this convergence is uniform.
For given \(\epsilon > 0\) choose \(N \in \N\) such that for all \(n, m \geq N\),
\begin{center}
    \(\ds \sup_{x \in E} \abs{f_n(x) - f_m(x)} \leq \epsilon\)
\end{center}
Then
\[
    \begin{aligned}
        \abs{f_n(x) - f(x)} & = \abs{f_n(x) - f_m(x) + f_m(x) - f(x)} \leq \abs{f_n(x) - f_m(x)} + \abs{f_m(x) - f(x)} \\
                            & \leq \abs{f_m(x) - f(x)} + \epsilon
    \end{aligned}
\]
Fix \(n \geq N\) and let \(m \ra \infty\). Observe that \(\abs{f_m(x) - f(x)} \ra 0\) due to pointwise convergence. Therefore for every \(x \in E\),
\[
    n \geq N \implies \abs{f_n(x) - f(x)} \leq \epsilon.
\]

코시 수열이 중요한 이유는 completeness 뿐만 아니라, 극한값을 알지 못할 때 수열의 수렴성을 논할 수 있기 때문입니다. 특히 급수의 경우 그 극한값을 알 수 없기 때문에 급수의 수렴판정 등에서 유용하게 사용됩니다.

\pagebreak
