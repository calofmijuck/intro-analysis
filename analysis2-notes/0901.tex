\section*{Introduction \& Notice}

\begin{itemize}
    \item 7, 8장 나가고 중간고사, 11장 나가고 기말고사
    \item 연습 시간이 있는 수업 (목 6:30 \(\sim\) 8:20)\footnote{가능하면 1시간 반 안에 끝내라고 하심 ㅋㅋ}
    \item 오늘 연습 시간: 지난학기 배운 내용 중 필요한 내용 복습
\end{itemize}

\chapter{Sequences and Series of Functions}

\section*{September 1st, 2022}

기본적으로 수열에 관련된 내용, real/complex-valued 수열이 아니라 함수가 주어졌을 때. 함수들을 모은 `\textbf{sequence of functions}'의 극한을 생각하는 것.

\medskip

Suppose \(E\) is a set\footnote{사실은 \textit{metric} space 이다.}, and let \(f_n: E \ra \C\). Then
\[
    \seq{f_n}_{n=1}^\infty
\]
is a sequence of (complex-valued) function.

\defn{7.1} \note{Pointwise Convergence} \(\seq{f_n}_{n = 1}^\infty\) converges \textbf{pointwise} on \(E\), if for each \(x \in E\) the sequence \(\seq{f_n(x)}_{n=1}^\infty\) converges in \(\C\).

In other words, for each \(x \in E\), there exists \(a_x \in \C\) and
\begin{center}
    \(\forall \epsilon > 0, \exists N \in \N\) such that \(n \geq N \implies \abs{f_n(x) - a_x} < \epsilon\).
\end{center}

\defn. If \(\seq{f_n}\) converges pointwise, we can define a function \(f\) by
\[
    f(x) = \lim_{n \ra \infty} f_n(x) \quad (x\in E)
\]

We say that
\begin{itemize}
    \item \(f\) is the \textit{limit} or \textit{limit function} of \(f_n\).
    \item \(\seq{f_n}\) to \(f\) pointwise on \(E\).
\end{itemize}

\defn. If \(\sum f_n(x)\) converges (pointwise) for every \(x \in E\), we can define
\[
    f(x) = \sum_{n=1}^\infty f_n(x) \quad (x \in E)
\]
and the function \(f\) is called the \textit{sum} of the series \(\sum f_n\).

\recall \(f: (E, d) \ra \C\) is continuous on \(E\) \miff \(f\) is continuous at all \(x \in E\).

\recall \note{Theorem 4.6} If \(p \in E\) and \(p\) is a limit point of \(E\),
\begin{center}
    \(f\) is continuous at \(p\) \miff \(\ds \lim_{x \ra p} f(x) = f(p)\)
\end{center}

\question Suppose \(\seq{f_n}\) is a sequence of functions. Does the limit function or the sum of the series preserve important properties?
\begin{enumerate}
    \item If \(f_n\) is continus, is \(f\) continuous?
    \item If \(f_n\) is differentiable/integrable, is \(f\) differentiable/integrable?
\end{enumerate}

For (1), the question is equivalent to the following:

\textit{If \(p\) is a limit point, does the following hold?}
\[
    \lim_{x\ra p} \lim_{n\ra \infty} f_n(x) \overset{?}{=} \lim_{n\ra \infty} \lim_{x\ra p}f_n(x)
\]

And the answer is \textbf{No}.

\ex{7.2} Suppose \(\ds a_{m, n} = \frac{m}{m + n}\) for \(m, n\in \N\). We see that
\[
    \lim_{n \ra \infty} \lim_{m \ra \infty} a_{m, n} = 1 \neq 0 = \lim_{m \ra \infty} \lim_{n \ra \infty} a_{m, n}
\]

\ex. Define
\[
    f_n(x) = \begin{cases}
        0       & \paren{\frac{1}{n} \leq x \leq 1} \\
        -nx + 1 & \paren{0 \leq x < \frac{1}{n}}
    \end{cases}
\]
then we can easily see that
\[
    f(x) = \lim_{n\ra \infty} f_n(x) = \begin{cases}
        0 & (0 < x \leq 1) \\
        1 & (x = 0)
    \end{cases}
\]
Thus \(f\) is not continuous at \(x = 0\).

% Example. Define \(f_n: \R \ra \R\).
% \[
%     f_n(x) = \frac{x^2}{(1+x^2)^n}\qquad (n = 0, 1, \dots)
% \] by direct calculation,
% \[
%     f(x) = \sum_{n = 0}^\infty f_n(x) = \sum_{n = 0}^\infty \frac{x^2}{(1 + x^2)^n} = 1 + x^2 \qquad (x \neq 0)
% \]
% If \(x = 0\), \(f(x) = 0\). Thus \(f\) is not continuous.

% Question. Suppose \(\seq{f_n}\) is a sequence of \textbf{Riemann integrable} functions that converges pointwise. Is \(f\) \textbf{Riemann integrable}? Also \textbf{No...}

% Example. For \(m = 1, 2, \dots\), define
% \[
% \begin{aligned}
%     f_m(x) &= \lim_{n\ra \infty} \left(\cos m! \pi x\right)^{2n} \\
%     &= \begin{cases}
%         1 & (m!x \in \Z) \\
%         0 & (m!x \notin \Z)
%     \end{cases}
% \end{aligned}
% \]

% Note that \(f_m(x) \in \mc{R}[a, b]\)

% Claim. \[
%     f(x) = \lim_{m\ra \infty} f_m(x) = \begin{cases}
%         1 & (x\in Q) \\
%         0 & (x \in \R \bs \Q)
%     \end{cases}
% \]
% and \(f(x)\) is nowhere continuous thus not Riemann integrable.

% \begin{pf}
%     Suppose \(x = p/q \in \Q\). (\(p, q\in \Z\)) If we take \(m \geq q\), we see that \(m! x \in \Z\). Thus \(f_m(x) = 1\).

%     If \(x \notin \Q\), \(m! x\) can never be in \(\Z\) and \(f_m(x) = 0\).
% \end{pf}

% 계속 예제...

% Question. Uniform continuity를 할 때 uniform이 어디서 나오죠? 해석학에서 그 점에서 뭐가 성립한다, 그러면 그 점과 그 근방에서만 확인하면 됐었죠. Continuity는 local property죠. 그런데 uniform continuity는 전체가 다 uniform하게 성립한다.

% Recall. \(f: (X, d) \ra (Y, d)\) is uniformly continuous on \(X\)\footnote{Subspace of metric space is also a metric space} if
% \[
%     \forall \epsilon > 0, \exists \delta > 0, d_X(q, p) < \delta \implies d_Y(f(p), f(q)) < \epsilon
% \]

% 모든 점에서 똑같이 잡을 수 있다!

% Fact. If \(X\) is compact and \(f\) is continuous on \(X\), then \(f\) is uniformly continuous on \(X\). (Theorem 4.19)\footnote{갑자기 왜 uniform continuity 얘기를 하냐, 헷갈리지 말고 기억하시라고!}

% 지금 나오는 uniform convergence는 sequence에 관한 것입니다!

% \begin{defn}[Uniform Convergence]
%     Suppose \(f_n: E \ra \C\) is a sequence of functions.

%     \(\seq{f_n}_{n=1}^\infty\) converges uniformly on \(E\) to a function \(f\) if
%     \[
%         \forall \epsilon > 0, \exists N : \forall x \in E, n \geq N, \abs(f_{n}(x) - f(x)) \leq \epsilon
%     \]\footnote{등호를 붙이는 것이 극한 잡기 편하다???}

%     \[
%         \forall \epsilon > 0, \sup_{x \in E} \abs{f_n(x) - f(x)} \leq \epsilon, \forall n \geq N
%     \]

%     \[
%         \sum_{n = 1}^\infty f_n
%     \] converges uniformly on \(E\): \(f(x) = \sum_{n=1}^\infty f_n(x)\) converges pointwise and \(\sum_{k=1}^n f_k(x)\) converges uniformly to \(f\).
% \end{defn}

% [똑같은 \(\epsilon\)-띠를 둘러서 \(y = f(x)\) 의 근방 안에 \(f_n(x)\) (\(n \geq N\)) 가 모두 들어가 있어야 한다]는 의미에서 uniform 이다.

% Notation. \(f_n \ra f\) uniformly on \(E\) \miff \(f_n \overset{u}{\ra} f\) on \(E\).\footnote{책에서는 나중에 \(\norm{f_n(x) - f(x)}_\infty \ra 0\) 으로 적었던 것 같은데...}

% 7.8에 나와있는 내용이 Cauchy sequence...

% Recall. Cauchy sequence converges!

% \begin{theorem}
%     (7.8) \(f_n \overset{u}{\ra} f\) on \(E\) \miff
%     \[
%         \forall \epsilon > 0, \exists N : \sup_{x \in E} \abs{f_n(x) - f_m(x)} \leq \epsilon, \forall n, \forall m \geq N
%     \]\footnote{Uniform Cauchy}
% \end{theorem}

% \begin{pf}
%     (\(\implies\)) For given \(\epsilon > 0\)...

%     Fix \(x \in E\),
%     \[\abs{f_n(x) - f_m(x)} \leq \abs{f_n(x) - f(x)} + \abs{f(x) - f_m(x)} \leq \frac{\epsilon}{2} + \frac{\epsilon}{2} = \epsilon\]
%     for \(n, m \geq N\).

%     (\(\impliedby\)) Uniform Cauchy property implies that \(\seq{f_n}\) is a Cauchy sequence in \(\C\). By the completeness of \(C\), the limit function \(f(x)\) exists. Now we show that this convergence is uniform.

%     For given \(\epsilon > 0\) choose \(N\) such that
%     \[\sup_{x\in E} \abs{f_n(x) - f_m(x)} \leq \epsilon\], for all \(n, m \geq N\). Then \[\abs{f_n(x) - f(x)} \leq \abs{\abs{f_n(x) - f_m(x)} - \abs{f_n(x) - f(x)}} + \abs{f_n(x) - f_m(x)} \leq \abs{f_m(x) - f(x)} + \epsilon\]

%     Fix \(n \geq N\) and let \(m \ra \infty\). Observe that the first term converges to 0 due to pointwise convergence.

%     \(\therefore \abs{f_n(x) - f(x)} \leq \epsilon, \forall n\geq N, \forall x \in E\)
% \end{pf}

\pagebreak
