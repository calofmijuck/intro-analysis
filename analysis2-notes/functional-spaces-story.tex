\subsection*{함수공간의 Story}

우리가 해석개론1에서 실수를 공부하기 위해서 어떻게 했었는지 떠올려 보면, 절대 실수 하나하나를 개별적으로 보지 않았습니다. 실수의 모임을 두고, 실수열을 공부하고, 위상적인 구조를 주는 등의 작업을 하고 나서야 실수를 제대로 이해할 수 있었습니다.

함수도 마찬가지입니다. 우리는 함수를 이해하기 위해서 함수 하나를 개별적으로 보는 것이 아니라, 함수가 속해있는 공간을 공부하는 것입니다. 실수를 공부할 때와 마찬가지로, 함수열을 공부하고, 위상적인 구조를 공부하며 함수공간을 이해하게 됩니다. 그런데 \(\R\)/\(\R^n\)과 함수공간의 가장 큰 차이점은 좌표공간은 유한차원이지만, \textbf{함수공간은 무한차원}이라는 점입니다.

함수공간을 벡터공간으로 만들기는 했으나, 선형대수학의 대부분 정리들은 벡터공간의 차원이 유한이라는 가정이 필요하기 때문에, 공짜로 얻어지는 정리는 없습니다. 이로 인해 \textbf{norm을 도입}하게 됩니다.

Norm을 도입하게 되면 공간에 거리 개념이 생기므로, metric space를 논할 수 있게 되고, 자연스럽게 수렴성을 논할 수 있게 됩니다. 이와 동시에 Cauchy 수열의 개념도 생겨납니다. 그리고 open ball을 정의할 수 있고, open/closed set이 정의되고, compact set까지 정의하게 됩니다. 함수공간에서는 norm이 없으면 아무것도 할 수 없습니다. \textbf{함수공간에는 norm이 항상 존재합니다.}

그리고 마지막으로 이 norm과 수렴 개념을 바탕으로 Cauchy 수열이 수렴하는지 살펴봅니다. 함수공간을 너무 작게 잡으면, Cauchy 수열이 수렴하지 않을 수 있습니다. 그러면 더 함수를 넣어야 합니다. 함수를 넣다보면 또 새로운 Cauchy 수열이 생깁니다. 이 과정을 반복하여 다 넣게 되면 드디어 Cauchy 수열이 수렴하게 되고, 비로소 \textbf{completeness}(완비)를 만족하게 됩니다.\footnote{Complete normed space를 Banach space라고 부릅니다.}

해석개론2의 첫 장에서는 \textbf{연속함수열의 수렴}에 대해 공부했습니다. 연속함수의 공간 \(C(X)\)에서 점별수렴과 고른수렴이 있었는데, 극한함수 또한 연속이어야 하기 때문에 우리는 이 함수공간에서 \textbf{고른수렴}을 올바른 수렴의 정의로 선택했습니다. 그래야 \(C(X)\)의 Cauchy 수열이 수렴하여 \(C(X)\)가 completeness를 만족하게 되기 때문입니다.

또 고른수렴에 대해서 공부하면서 얻은 부산물로, 고른수렴이 언제 미분가능성과 적분가능성을 보존하는지 공부했습니다. 극한함수의 미분가능성에 대해서는 굉장히 까다로운 조건이 필요했지만, 적분가능성의 경우 잘 보존되는 것을 확인했습니다. 이를 기점으로 해석학은 적분에 주안점을 두고 가게 됩니다. 우리가 함수를 미분하면 함수가 나빠지는 반면, 적분을 통해 얻은 함수는 상대적으로 다루기 쉽습니다. 또한 현실 세계에서 미분가능한 함수를 만나기 쉽지 않기도 합니다. \textbf{결국 해석학은 적분을 발전시키는 방향으로 나아가게 됩니다.}

수렴을 공부한 이후, 본격적으로 함수공간 \(C(X)\)를 공부했습니다. 이제 우리의 관심사는 \(C(X)\)의 compact set 입니다. 따라서 \textbf{수렴하는 부분수열}에 대해 공부하게 됩니다.\footnote{당연히 \(C(X)\)의 수렴은 고른수렴입니다.} 이는 마치 \(R^n\)에서 Bolzano-Weierstrass 정리를 공부했던 것과 동일합니다. 그래서 \(C(X)\)의 수렴하는 부분수열을 찾기 위해 점별유계, 고른유계, 동등연속의 개념을 공부했으며 Arzela-Ascoli 정리를 공부했습니다. 그리고 실제로 주어진 연속함수로 수렴하는 연속함수열(특별히 다항함수)이 존재한다는 사실을 Weierstrass 정리를 통해 공부했습니다. 이는 곧 \(\Q\)가 \(\R\)에서 조밀(dense)했던 것처럼 다항함수가 \(C(X)\)에서 조밀함을 보여줍니다.

수렴하는 부분수열이 중요한 또 다른 이유는 \textbf{precompact} 개념 때문이기도 합니다. 집합 \(X\)의 수열이 수렴하는 부분수열을 가질 때, \(X\)를 precompact set이라고 합니다. Precompact 개념을 이용하면 \(\seq{f_n}\)이 \(f\)로 수렴하는지 확인하려 할 때, 이를 2단계로 나눠 증명할 수 있게 됩니다. 먼저 \(\{f_n : n \in \N\}\) 이 precompact임을 보이고, \(\seq{f_n}\)의 부분수열이 \(g\)로 수렴하면 \(f = g\) 임을 보이면 됩니다.

이후로는 수열공간 \(\ell^p(\N)\), 특이적분가능함수공간 \(\mc{R}^p(I)\)에 대해 공부하게 되는데, 여기서도 같은 story가 반복됩니다. 함수공간을 벡터공간으로 만들어 norm을 정의하고, Cauchy 수열이 수렴하는지 확인하여 completeness를 확인합니다. 그리고 compact set에 대해 조사하게 됩니다. 수열공간 \(\ell^p(\N)\)의 경우 유계이고 닫힌 집합만으로 compact set이 되기에는 부족함을 확인하게 되고,\footnote{실제로, 무한차원의 normed space는 Heine-Borel property (compact \miff bounded \& closed)를 가질 수 없음이 알려져 있습니다.} 특이적분가능함수공간 \(\mc{R}^p(I)\)는 complete 하지 않다는 것도 확인하게 됩니다. 이는 적분가능함수공간을 너무 작게 잡았다는 의미로, 후에 르벡 적분 등장의 배경이 되며 르벡적분가능함수공간 \(L^p(E)\)는 completeness를 만족하게 됩니다.

\pagebreak
