\section*{September 13th, 2022}

Equicontinuous 하기 전에 7.23을 먼저 할게요. 7.23을 하기 위한 motivation이 필요합니다.

\recall \note{Bolzano-Weierstrass} Suppose \(\seq{a_n}\) is a sequence in \(\C\). If \(\seq{a_n}\) is bounded, then there exists a convergent subsequence of \(\seq{a_n}\).

\recall A sequence \(\seq{a_n}\) in a metric space \((E, d)\) is bounded if
\begin{center}
    \(\exists x_0 \in E\) and \(\exists r > 0\) such that \(a_n \in B_r(x_0)\) for all \(n \in \N\).
\end{center}

\defn. \note{Pointwise Bounded} Suppose \(f_n: E \ra \C\). We say that \(\seq{f_n}\) is \textbf{pointwise bounded} on \(E\) if the sequence \(\seq{f_n(x)}_{n=1}^\infty\) is bounded for all \(x \in E\). Or equivalently,
\begin{center}
    \(\ds \sup_{n \in \N} \abs{f_n(x)} < \infty\) for each \(x \in E\).
\end{center}

\thm{7.23} Suppose \(E\) is at most countable.\footnote{Countable 이거나 그거보다 작아서 finite 이거나. Countable이 아니면 안돼요.} Suppose \(f_n: E \ra \C\) and \(\seq{f_n}\) is pointwise bounded. Then there exists a subsequence \(\seq{f_{n_k}}\) of \(\seq{f_n}\) such that \(\seq{f_{n_k}(x)}\) converges for all \(x \in E\).
\begin{center}
    \(\exists \seq{n_k}\) such that \(f(x) = \ds \lim_{k \ra\infty} f_{n_k}(x)\) for each \(x \in E\). % \(\exists f: E \ra \C\) such that
\end{center}

\pf \textit{Diagonalization!} Let \(E = \seq{x_i}_{i=1}^\infty\). (countable) Since \(\seq{f_n(x_1)}\) is bounded, there exists a convergent subsequence by Bolzano-Weierstrass Theorem. Let the subsequence be \(\seq{f_{1, k}}_{k=1}^\infty\). Similarly, since \(\seq{f_{1, k}(x_2)}\) is bounded, there exists a convergent subsequence \(\seq{f_{2, k}}_{k=1}^\infty\) of \(\seq{f_{1, k}}_{k=1}^\infty\). Since \(\seq{f_{n, k}}\) is pointwise bounded, we can repeat this procedure to get the next subsequence \(\seq{f_{n+1, k}}_{k=1}^\infty\).

Now we arrange the sequence,
\[
    \begin{matrix}
        f_{1, 1} & f_{1, 2} & f_{1, 3} & f_{1, 4} & \cdots \\
        f_{2, 1} & f_{2, 2} & f_{2, 3} & f_{2, 4} & \cdots \\
        f_{3, 1} & f_{3, 2} & f_{3, 3} & f_{3, 4} & \cdots \\
        f_{4, 1} & f_{4, 2} & f_{4, 3} & f_{4, 4} & \cdots \\
        \vdots   & \vdots   & \vdots   & \vdots   & \ddots
    \end{matrix}
\]
and consider the diagonal elements. Then \(\ds \lim_{n \ra \infty} f_{n, n}(x)\) converges for all \(x \in E\). This is because \(\seq{f_{n+1, k}}\) is always a subsequence of \(\seq{f_{n, k}}\), and \(\seq{f_{n, k}}\) was chosen to converge on \(\{x_1, x_2, \dots, x_n\}\).

\defn{7.19} Suppose \(f_n: E \ra \C\).
\begin{enumerate}
    \item \note{Pointwise Bounded} \(\seq{f_n}\) is \textbf{pointwise bounded} on \(E\) if there exists a finite valued function \(\varphi: E \ra \R^+\) such that
          \begin{center}
              \(\abs{f_n(x)} \leq \varphi(x) < \infty\) for all \(x\in E\).
          \end{center}
    \item \note{Uniformly Bounded} \(\seq{f_n}\) is \textbf{uniformly bounded} on \(E\) if there exists \(M > 0\) such that
          \begin{center}
              \(\abs{f_n(x)} < M\), \(\forall x \in E\) and \(n \geq 1\).
          \end{center}
          Or equivalently,
          \begin{center}
              \(\ds \sup_{n\in \N} \sup_{x \in E} \abs{f_n(x)} < \infty\)
          \end{center}
\end{enumerate}

\rmk
\begin{enumerate}
    \item Uniform boundedness implies pointwise boundedness, but the converse is false.
    \item Every uniformly convergent sequence of bounded functions is uniformly bounded.
\end{enumerate}

\medskip

\question Let \(E\) be a compact metric space.
\begin{enumerate}
    \item \textit{Suppose that \(\seq{f_n}_{n=1}^\infty\) is uniformly bounded or continuous. Is there a pointwise convergent subsequence \(\seq{f_{n_k}}\) of \(\seq{f_n}\)?} \textbf{No}.
    \item \textit{Suppose that \(\seq{f_n}\) converges pointwise and is uniformly bounded. Is there a uniformly convergent subsequence \(\seq{f_{n_k}}\) of \(\seq{f_n}\)?} \textbf{No}.
\end{enumerate}

아래 2개의 예시는 위 질문에 대한 대답이 `아니오'임을 알려줍니다.

\ex{7.20} Let
\[
    f_n(x) = \sin nx, \quad (x \in [0, 2\pi]).
\]
\(f_n\) is uniformly bounded (\(\abs{\sin nx} \leq 1\)). Suppose there exists a subsequence \(f_{n_k}\) such that \(\seq{f_{n_k}(x)}\) converges for all \(x \in [0, 2\pi]\). Then as \(k\ra\infty\), \(f_{n_k}(x) - f_{n_{k+1}}(x) \ra 0\) for all \(x \in [0, 2\pi]\).

However, as \(k \ra \infty\),
\[
    \int_0^{2\pi} (f_{n_k} - f_{n_{k+1}})^2 \d{x} \ra \int_0^{2\pi} 0^2 \d{x} = 0
\]
by Theorem 11.32.\footnote{Lebesgue's theorem concerning integration of boundedly convergent sequences.} But simple calculation shows that
\[
    \int_0^{2\pi} (f_{n_k} - f_{n_{k+1}})^2 \d{x} = 2\pi,
\]
which leads to a contradiction. Thus such subsequence cannot exist.

\ex{7.21} Let
\[
    f_n(x) = \frac{x^2}{x^2 + (1-nx)^2}, \quad (x\in [0, 1]).
\]
\(f_n\) is uniformly bounded (by 1), and \(f_n \ra 0\) pointwise. Also, note that \(f_n(1/n) = 1\).

Suppose there exists a subsequence \(\seq{f_{n_k}}\) such that \(f_{n_k} \uc 0\). Then, \(\exists k_0 \in \N\) such that
\[ \tag{\mast}
    \sup_{x \in [0, 1]} \abs{f_{n_{k_0}}(x)} < \frac{1}{2}.
\]
However,
\[
    1 = \abs{f_{n_{k_0}}\left(\frac{1}{n_{k_0}}\right)} \leq \sup_{x \in [0, 1]} \abs{f_{n_{k_0}}(x)} < \frac{1}{2},
\]
leading to a contradiction. Thus such subsequence cannot exist.

\defn{7.22} \note{Equicontinuity} Let \((X, d)\) be a metric space, and \(E \subset X\). Let \(\scr{F}\) be a family of complex-valued functions on \(E\). We say that \(\scr{F}\) is \textbf{equicontinuous} on \(E\) if
\begin{center}
    \(\forall \epsilon > 0, \exists \delta > 0\) such that \(d(x, y) < \delta \implies \abs{f(x) - f(y)} < \epsilon\) for all \(x, y \in E, f\in \scr{F}\).
\end{center}
That is, all \(f \in \scr{F}\) are uniformly continuous on \(E\) with the same \((\epsilon, \delta)\) in the definition.

\(f\)가 하나라면 그냥 uniform continuity의 정의입니다. 그런데 \(f\)가 family of functions이고, 이러한 \(\epsilon\)을 모든 \(f\)에 대해서 잡을 수 있다는 의미에서 `equi' 입니다.

\rmk \note{Example 7.21} Let \(\scr{F} = \seq{f_n}\). Then \(\scr{F}\) is not equicontinuous on \(E\).

\pf Suppose \(\scr{F}\) is equicontinuous on \(E\). For \(\epsilon = 1\), there should exist \(\delta > 0\) such that
\begin{center}
    \(\forall x, y \in [0, 1]\) and \(\abs{x - y} < \delta \implies \abs{f_n(x) - f_n(y)} < 1\).
\end{center}
In particular, \(\abs{f_n(0) - f_n(1/n)} < 1\) if \(1 / n < \delta\), but \(\abs{f_n(0) - f_n(1/n)} = 1\). Contradiction.

\medskip

\thm{7.24} Let \(K\) be a compact set. Suppose \(f_n \in C(K)\). If \(f_n\) converges uniformly on \(K\), \(\mathscr{F} = \seq{f_n}\) is equicontinuous on \(K\).

\pf Let \(\epsilon > 0\) be given. Since \(\seq{f_n}\) is uniformly Cauchy, \(\exists N\in \N\) such that for all \(x \in E\),
\begin{center}
    \(n\geq N \implies \abs{f_n(x)  - f_N(x)} < \dfrac{\epsilon}{3}\).
\end{center}
Since continuous function defined on a compact set is uniformly continuous,\footnote{Heine-Cantor Theorem.} \(f_n\) is uniformly continuous. Therefore \(\exists \delta > 0\) such that
\begin{center}
    \(d(x, y) < \delta \implies \abs{f_N(x) - f_N(y)} < \dfrac{\epsilon}{3}\) for \(x, y \in K\).
\end{center}
For \(x, y\in K\), if \(n \geq N\) and \(d(x, y) < \delta\),
\[
    \abs{f_n(x) - f_n(y)} \leq \abs{f_n(x) - f_N(x)} + \abs{f_N(x) - f_N(y)} + \abs{f_N(y) - f_n(y)} < \frac{\epsilon}{3} + \frac{\epsilon}{3} + \frac{\epsilon}{3} = \epsilon.
\]
Therefore \(\seq{f_n}_{n=N}^\infty\) is equicontinuous. Additionally, \(\seq{f_n}_{n=1}^N\), which is a finite union of uniformly continuous functions, is equicontinuous. Thus \(\seq{f_n}\) is equicontinuous because it is a union of two equicontinuous family of functions.\footnote{유한개는 equicontinuous 로 누르고, 꼬리는 uniform convergence로 누르고.}

\pagebreak
