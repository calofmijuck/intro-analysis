\section*{September 6th, 2022}

More examples.

\ex{7.5} Consider \(f_n(x) = \dfrac{\sin nx}{\sqrt{n}}\) for \(x \in \R\).
\[
    f(x) = \lim_{n\ra\infty} f_n(x) \equiv 0
\]
but,
\[
    f'_n(x) = \sqrt{n}\cos nx \implies f'_n(0) = \sqrt{n}.
\]
As \(n \ra \infty\), \(f'_n(0)\) does not converge to \(f'(0)\).

\ex{7.6} Consider \(f_n(x) = nx(1-x^2)^n\) for \(x \leq 0 \leq 1\). Note that
\begin{center}
    \(f_n(0) = 0\), \(f_n(1) = 0\).
\end{center}
When \(0 < x < 1\), \(f_n \ra f \equiv 0\). \note{Theorem 3.20 (d)} Thus \(\ds \lim_{n\ra \infty} f_n(x) = 0\) for \(0 \leq x \leq 1\).

But
\[
    \int_0^1 nx(1-x^2)^n \d{x} = \left[\frac{-n}{2n+2} (1-x^2)^{n+1}\right]_0^1 = \frac{n}{2n+2},
\]
and thus
\[
    \lim_{n\ra \infty} \int_0^1 f_n(x) \d{x} = \frac{1}{2} \neq 0 = \int_0^1 f(x)\d{x}.
\]

\defn. \(\ds \sum_{n=1}^\infty f_n\) converges uniformly on \(E\) \miff \(\ds \seq{\sum_{k=1}^n f_k}\) converges uniformly on \(E\).

\thm{7.10} \note{Weierstrass \(M\)-test} Suppose \(f_n: E \ra \C\) and that for every \(n\), \(\exists M_n \in \R\) such that
\[
    \abs{f_n(x)} \leq M_n, \quad (x \in E)
\]
and \(\sum_{n=1}^\infty M_n < \infty\). Then the series \(\sum_{n=1}^\infty f_n\) converges uniformly on \(E\).

\pf We want to show that the series is Cauchy.\\
For \(m > n\), we have
\[
    \abs{\sum_{k=n}^m f_k(x)} \leq \sum_{k=n}^m \abs{f_k(x)} \leq \sum_{k=n}^m M_k.
\]
Given \(\epsilon > 0\), choose \(m, n \in \N\) such that for \(m, n \geq N\), \(\sum_{k = n}^m M_k < \epsilon\). Then we get
\begin{center}
    \(\ds \abs{\sum_{k=n}^m f_k(x)} < \epsilon\), for all \(m, n\geq N\).
\end{center}
By Theorem 7.8, \(\sum f_n\) converges uniformly.

\thm{7.11} Given metric space \((Y, d)\) and \(E \subset Y\), suppose that \(f_n \uc f\) on \(E\) and \(x \in E'\). If
\begin{center}
    \(\ds \lim_{t \ra x} f_n(t) = A_n \in \C\), \quad (limit exists)
\end{center}
then the sequence \(\seq{A_n}\) converges, and
\begin{center}
    \(\ds \lim_{n \ra \infty} A_n = \lim_{t \ra x} f(t)\).
\end{center}
In conclusion,
\[
    \lim_{n\ra \infty} \lim_{t \ra x} f_n(t) = \lim_{t\ra x}\lim_{n\ra\infty} f_n(t).
\]

\pf\\
\note{\(\seq{A_n}\) converges in \(\C\)} Since \(\C\) is complete, we will show that \(\seq{A_n}\) is a Cauchy sequence. Let \(\epsilon > 0\). Since \(f_n \uc f\) on \(E\),
\begin{center}
    \(\exists N \in \N\) such that \(n, m\geq N \implies \abs{f_n(t) - f_m(t)} \leq \epsilon\). (\(\forall t\in E\))
\end{center}
From \(\ds \lim_{t \ra x} f_n(t) = A_n\), we can choose \(t\) arbitrarily close to \(x\), such that for \(n, m \geq N\),
\begin{center}
    \(\abs{f_n(t) - A_n} < \epsilon\) and \(\abs{f_m(t) - A_m} < \epsilon\).
\end{center}
Therefore for all \(n, m \geq N\),
\[
    \begin{aligned}
        \abs{A_n - A_m} &= \abs{A_n - f_n(t) + f_n(t) - A_m + f_m(t) - f_m(t)}\\
        &\leq \abs{f_n(t) - A_n} + \abs{f_m(t) - A_m} + \abs{f_n(t) - f_m(t)} < 3\epsilon,
    \end{aligned}
\]
and thus \(\seq{A_n}\) is a Cauchy Sequence.

\note{\(\ds \lim_{n \ra \infty} A_n = \lim_{n \ra x} f(t)\)} Let \(A = \ds \lim_{n \ra \infty} A_n\). We want to show that for all \(\epsilon > 0\),
\begin{center}
    \(\exists \delta > 0\) such that \(0 < d(t, x) < \delta \implies \abs{f(t) - A} < \epsilon\).
\end{center}
Now,
\begin{center}
    \(\ds \abs{f(t) - A} \leq \sup_{s \in E} \abs{f(s) - f_n(s)} + \abs{f_n(t) - A_n} + \abs{A_n - A}\).
\end{center}
Given \(\epsilon > 0\), choose \(N \in \N\) such that for \(n \geq N\),
\begin{center}
    \(\ds \sup_{s\in E} \abs{f(s) - f_n(s)} < \frac{\epsilon}{3}\) and \(\ds \abs{A_n - A} < \frac{\epsilon}{3}\).
\end{center}
Fix such \(N\) and choose \(\delta\) such that for \(0 < d(x, t) < \delta\) and \(t \in E\),
\[
    \abs{f_N(t) - A_N} \leq\frac{\epsilon}{3}.
\]
Thus for \(t\in E\) and \(0 < d(x, t) < \delta\),
\[
    \abs{f(t) - A} \leq \frac{\epsilon}{3} + \frac{\epsilon}{3} + \frac{\epsilon}{3} = \epsilon.
\]

\thm{7.12} Suppose \(f_: E \ra \C\) is continuous on \(E\) and \(f_n \uc f\) on \(E\). Then \(f\) is continuous on \(E\).

\pf Let \(x \in E\). If \(x \in E'\), \(f\) is continuous at \(x\) by Theorem 7.11. If \(x\) is an isolated point (not a limit point), \(f\) is continuous at \(x\) by definition of continuity.

\textbf{앞으로는 \(E\)를 전부 metric space라고 가정할게요.}

이 정리는 언제 uniformly converge 하는지 알려줍니다.

\thm{7.13} Given a compact metric space \(K\), suppose that
\begin{enumerate}
    \item \(f_n\) and \(f: K \ra \C\) are continuous on \(K\).
    \item \(f_n \ra f\) pointwise.
    \item \(f_n(x) \geq f_{n+1}(x)\) for \(x \in K\).\footnote{\(f_n\) only needs to be monotone. See Dini's Theorem.}
\end{enumerate}
Then \(f_n \uc f\) on \(K\).

\pf Let \(g_n(x) = f_n(x) - f(x)\). Then \(g_n(x)\) is continuous, decreasing and \(g_n \ra 0\) as \(n \ra \infty\) for all \(x \in K\). Let \(\epsilon > 0\) be given.

\quad \claim. There exists \(N \in \N\) such that \(0 \leq g_n(x) < \epsilon\) for all \(x \in K\).

\quad \pf Let \(K_n = \{x \in K : g_n(x)\geq \epsilon\}\). Then \(K_n = K \cap g_n\inv ([\epsilon, \infty))\).\footnote{Closed subset of a compact set is also compact, and the inverse image of closed set is closed if the function is continuous} Since \(g_n\) is decreasing, \(K_{n+1}\subset K_n\), but because \(g_n \ra 0\), \(\bigcap_{n=1}^\infty K_n = \varnothing\). By Theorem 2.36, there exists \(N \in \N\) such that \(K_N = \varnothing\), and then \(K_n = \varnothing\) for \(n \geq N\). Thus, \(0 \leq g_n(x) < \epsilon\) for \(\forall x \in K\), \(\forall n \geq N\).

\rmk Compactness is necessary here. Consider \(f_n(x) = \dfrac{1}{nx + 1}\) on \(x \in E = (0, 1)\). \(f_n\) does not converge to \(0\) uniformly.

\pf Suppose \(f_n \uc 0\), and take \(\epsilon = 1/2\). Then,
\begin{center}
    \(\exists N \in \N\) such that \(\ds x \in (0, 1) \implies \frac{1}{Nx + 1} < \frac{1}{2}\).
\end{center}
This gives a contradiction because the equation above gives \(Nx > 1\), but we can choose \(x\) arbitrarily close to \(0\).

\defn. Let \((X, d)\) be a metric space. Define
\[
    C(X, \C) = \{f: X \ra \C \mid f\text{ is continuous and bounded}\}.
\]
If there is no ambiguity, we write \(C(X) = C(X, \C)\).

Let \(\norm{f} = \ds \sup_{x\in X} \abs{f(x)}\). Then \(\norm{\cdot}\) is a norm on \(C(X)\).
\begin{enumerate}
    \item \(\norm{f} = 0 \iff f \equiv 0\).
    \item \(\norm{f} < \infty\).
    \item \(\norm{f + g} \leq \norm{f} + \norm{g}\).
\end{enumerate}
Define \(d(f, g) = \norm{f - g}\), then \((C(X), d)\) is a metric space.

Therefore, \(f_n \uc f \iff f_n \ra f\) in \((C(X), d)\).
