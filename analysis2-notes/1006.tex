\section*{October 6th, 2022}

When does the Fourier series converge? Can continuous functions be expressed as a trigonometric series?

\defn. \note{Dirichlet Kernel} Let \(x \in [-\frac{\pi}{2}, \frac{\pi}{2}]\).\footnote{이 조건은 왜 필요하지?}
\[
    D_N(x) = \sum_{n = -N}^N e^{inx} = \begin{cases}
        \dfrac{\sin\left(N + \frac{1}{2}\right)x}{\sin\frac{x}{2}} & (x\neq 0) \\
        2N + 1                                                     & (x = 0)
    \end{cases}.
\]

The last expression comes from the fact that
\[
    (e^{ix} - 1) D_N(x) = e^{i(N+1)x} - e^{-iNx},
\]
when \(D_N(x)\) is viewed as a geometric series.

Therefore,
\[
    \begin{aligned}
        s_N(f; x) = \sum_{n=-N}^N c_n e^{inx} & = \sum_{n = -N}^N \frac{1}{2\pi}\int_{-\pi}^\pi f(t)e^{-int} \d{t} \cdot e^{inx} \\
                                              & = \sum_{n = -N}^N \frac{1}{2\pi}\int_{-\pi}^\pi f(t)e^{in(x - t)} \d{t}          \\
                                              & = \sum_{n = -N}^N \int_{-\pi}^\pi f(t) D_N(x-t) \d{t}                            \\
                                              & = \sum_{n = -N}^N \int_{-\pi}^\pi f(x - u) D_N(u) \d{u}, \quad (u = x - t)
    \end{aligned}
\]
where the integration bounds does not change since \(f\) is periodic with period \(2\pi\).

\thm{8.14} \footnote{Note that we have the assumption: \(f\) is periodic and \(f \in \mc{R}\).} For some \(x \in \R\), if there exists \(M > 0\) and \(\delta > 0\) such that
\[
    \abs{f(x + t) - f(x)} \leq M\abs{t}
\]
for all \(\abs{t} < \delta\), then
\[
    \lim_{N \ra\infty} s_N(f; x) = f(x).
\]

\pf Let \(x\) be fixed. Note that
\[
    \frac{1}{2\pi} \int_{-\pi}^\pi D_N(t)\d{t} = 1.
\]
We want \(s_N(f;x) - f(x)\) to be small enough. So,
\[
    \begin{aligned}
        s_N(f; x) - f(x) & = \frac{1}{2\pi} \int_{-\pi}^\pi \big(f(x-t) - f(x)\big)D_N(t)\d{t}                                                               \\
                         & \overset{(?)}{=} \frac{1}{2\pi} \int_{-\pi}^\pi \frac{f(x-t) - f(x)}{\sin \frac{t}{2}} \sin \left(N + \frac{1}{2}\right) t \d{t}.
    \end{aligned}
\]
But \(\sin\frac{t}{2}\) is undefined for \(t = 0\), so we define
\[
    g(t) = \begin{cases}
        \dfrac{f(x-t) - f(x)}{\sin \frac{t}{2}} & (0 < \abs{t} \leq \pi) \\
        0                                       & (t = 0)
    \end{cases},
\]
and write
\[
    \begin{aligned}
        s_N(f; x) - f(x) & = \frac{1}{2\pi} \int_{-\pi}^\pi g(t) \sin \left(N + \frac{1}{2}\right) t \d{t}                                                                     \\
                         & = \frac{1}{2\pi} \int_{-\pi}^\pi g(t)\cos\frac{t}{2} \sin Nt \d{t} + \frac{1}{2\pi} \int_{-\pi}^\pi g(t)\sin\frac{t}{2} \cos Nt \d{t}. \quad (\ast)
    \end{aligned}
\]

Recall that \(\{\phi_0, \phi_{2n}, \phi_{2n-1}\}\) where
\[
    \phi_0 = \frac{1}{\sqrt{2\pi}}, \quad \phi_{2n} = \frac{\sin nt}{\sqrt{\pi}}, \quad \phi_{2n-1} = \frac{\cos nt}{\sqrt{\pi}}
\]
is an orthonormal system. Since \(g(t)\cos\frac{t}{2}\) and \(g(t) \sin \frac{t}{2}\) are Riemann integrable (bounded and continuous except for \(t = 0\)), \((\ast) = \widehat{c_{2N}}(h_1) + \widehat{c_{2N-1}}(h_2)\) where \(\widehat{c_{2N}}, \widehat{c_{2N-1}}\) are Fourier coefficients of
\[
    h_1(t) = \frac{\sqrt{\pi}}{2\pi} g(t)\cos \frac{t}{2}, \quad h_2(t) = \frac{\sqrt{\pi}}{2\pi} g(t)\sin \frac{t}{2},
\]
relative to \(\{\phi_n\}\). By {\sffamily Theorem 8.12},
\[
    s_N(f; x) - f(x) = \widehat{c_{2N}}(h_1) + \widehat{c_{2N-1}}(h_2) \ra 0
\]
as \(N \ra \infty\).

\thm. Suppose \(f\) is continuous. Then for all \(\epsilon > 0\), there exists a trigonometric polynomial \(P\) such that,
\[
    \abs{P(x) - f(x)} < \epsilon, \quad (x \in \R).
\]

\rmk The statement is equivalent to:
\begin{center}
    \(\exists\) a sequence of trigonometric polynomials \(P_n\) such that \(\ds \sup_{x\in \R} \abs{P_n(x) - f(x)} \ra 0\) as \(n \ra \infty\).
\end{center}

\pf We consider the domain of \(f\) as \([0, 2\pi)\), and define
\[
    T = \{(a, b) \in \R^2 : a^2 + b^2 = 1\} \approx \{e^{ix} : x \in [0, 2\pi)\}.
\]
Then \(h(x) = e^{ix}\) is a bijection from \([0, 2\pi)\) to \(T\), so \(h\inv\) exists. Now let
\[
    g = f\circ h\inv
\]
so that \(f(x) = g(e^{ix})\). Then \(g : T \ra \C\), and \(T\) is compact in \(\R^2\). Let
\[
    \mc{A} = \left\{ \sum_{n=-N}^N c_n z^n : z\in T, c_n\in \C \right\}.
\]
Then \(\mc{A}\) is a self-adjoint algebra.\footnote{Check! The `magic bars' will do the work.} Using the identity function of \(T\), we see that \(\mc{A}\) separates points of \(T\) and \(\mc{A}\) vanishes at no point of \(T\). By {\sffamily Theorem 7.33}, \(\mc{A}\) is dense in \(C(T, \C)\). Therefore, for all \(\epsilon > 0\), there exists
\begin{center}
    \(\widehat{p} \in \mc{A}\) such that \(\ds \sup_{z \in T} \abs{\widehat{p}(z) - g(z)} < \epsilon\).
\end{center}
Using \(p(x) = \widehat{p}(e^{ix})\), we recover \(f(x)\) and get
\[
    \sup_{x \in [0, 2\pi)} \abs{p(x) - g(h(x))} = \sup_{x \in [0, 2\pi)} \abs{p(x) - f(x)} < \epsilon.
\]

\medskip

Our last question is, what if \(f\) is not continuous?

\thm{8.16} \note{Parseval} Suppose that \(f, g \in \mc{R}^{2}[-\pi, \pi]\) and periodic with period \(2\pi\). If
\[
    f \sim \sum_{n=-\infty}^\infty c_n e^{inx}, \quad g \sim \sum_{n=-\infty}^\infty \gamma_n e^{inx},
\]
the following holds.
\begin{enumerate}
    \item \(\ds \lim_{N \ra \infty} \frac{1}{2\pi} \int_{-\pi}^\pi \abs{f(x) - s_N(f; x)}^2 \d{x} = 0\).
    \item \note{Parseval's Identity} \(\ds \frac{1}{2\pi} \int_{-\pi}^\pi f(x) \overline{g(x)} \d{x} = \sum_{n = -\infty}^\infty c_n \overline{\gamma_n}\).
    \item \(\ds \frac{1}{2\pi} \int_{-\pi}^\pi \abs{f(x)}^2 \d{x} = \sum_{n = -\infty}^\infty \abs{c_n}^2\). (Directly follows from (2))
\end{enumerate}

\pf We will use the notation
\[
    \norm{h}_2 = \left\{\frac{1}{2\pi} \int_{-\pi}^\pi \abs{h(x)}^2 \d{x}\right\}^{1/2}.
\]
(1) Let \(\epsilon > 0\) be fixed. From {\sffamily Problem 6.12},\footnote{\(f \in \mc{R}\) 이면 \(f\)를 근사하는 연속함수 \(h\)를 잡을 수 있다. (과제 \#2)} there exists a continuous function \(h\) with period \(2\pi\) such that
\[
    \norm{f - h}_2 < \frac{\epsilon}{3}.
\]
By {\sffamily Theorem 8.15}, there exists a trigonometric polynomial \(p\) such that
\[
    \norm{h - p}_2 < \frac{\epsilon}{3}.
\]
We observe the degree of \(p\), and let \(N_0 = \deg p\). Then by {\sffamily Theorem 8.11},\footnote{Degree를 높이면 근사가 더 정확해진다.}
\[
    \norm{h - s_N(h)}_2 \leq \norm{h - p}_2 < \frac{\epsilon}{3}
\]
holds for \(N \geq N_0\). Finally the following holds, \(\norm{s_N(h-f)}_2 \leq \norm{h - f}_2 < \epsilon\).\footnote{\(s_N(f;x)\)의 정의 참고. \[\int_{-\pi}^\pi \abs{s_N(f;x)}^2 \d{x} \leq \int_{-\pi}^\pi \abs{f(x)}^2 \d{x}\]}

Now we use {\sffamily Problem 6.11}. (Triangle Inequality for Norms)
\[
    \begin{aligned}
        \norm{f - s_N(f)}_2 & \leq \norm{f - h}_2 + \norm{h - s_N(h)}_2 + \norm{s_N(h) - s_N(f)}_2                                                                    \\
                            & = \norm{f - h}_2 + \norm{h - s_N(h)}_2 + \norm{s_N(h - f)}_2 < \frac{\epsilon}{3} + \frac{\epsilon}{3} + \frac{\epsilon}{3} = \epsilon.
    \end{aligned}
\]

(2) By {\sffamily Problem 6.10}, (Schwarz Inequality for Integrals)
\[
    \abs{\int_{-\pi}^\pi f \overline{g} - \int_{-\pi}^\pi s_N(f) \overline{g}} \leq \int_{-\pi}^\pi \abs{f - s_N(f)}\abs{g} \leq \left(\int_{-\pi}^\pi \abs{f - s_N(f)}^2\right)^{1/2} \left(\int_{-\pi}^\pi \abs{g}^2\right)^{1/2} \ra 0
\]
as \(N \ra \infty\), by (1). Now we have
\[
    \begin{aligned}
        \frac{1}{2\pi} \int_{-\pi}^\pi s_N(f) \overline{g(x)}\d{x} & = \sum_{n = -N}^N c_n \frac{1}{2\pi} \int_{-\pi}^\pi e^{inx} \overline{g(x)}\d{x}                                            \\
                                                                   & =\sum_{n = -N}^N c_n \frac{1}{2\pi} \overline{\int_{-\pi}^\pi e^{-inx} g(x)\d{x}} = \sum_{n = -N}^N c_n \overline{\gamma_n},
    \end{aligned}
\]
which converges to \(\ds \sum_{n=-\infty}^\infty c_n \overline{\gamma_n}\).

\pagebreak
