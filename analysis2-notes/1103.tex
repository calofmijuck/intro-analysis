\section*{November 3rd, 2022}

\(f_n \geq 0\) 일 때 성립하는 것부터 살펴봅시다.

\thm{11.31} \note{Fatou} If \(f_n \geq 0\) and measurable and \(E\) is measurable. Then
\[
    \int_E \liminf_{n\ra\infty} f_n \d{\mu} \leq \liminf_{n\ra\infty} \int_E f_n \d{\mu}.\footnote{ILLLI: integral of limit less than limit of integral.}
\]

\pf Let \(E = X\). Let \(g_n = \ds \inf_{k \geq n} f_k\). Then \(\ds \lim_{n \ra \infty} g_n = \liminf_{n\ra\infty} f_n\), and we know that \(g_n\) is increasing and non-negative. By definition, \(g_n \leq f_k\) for all \(k \geq n\). Therefore,
\[
    \int g_n \d{\mu} \leq \inf_{k\geq n} \int f_k \d{\mu}.
\]
Letting \(n \ra \infty\) gives
\[
    \int \liminf_{n\ra\infty} f_n \d{\mu} = \lim_{n \ra \infty} \int g_n \d{\mu} \leq \lim_{n \ra \infty} \inf_{k \geq n}\int f_k \d{\mu} = \liminf_{n \ra \infty} \int f_n \d{\mu},
\]
where the first equality holds by the MCT.

\rmk The above theorem doesn't work for \(\limsup\).
\[
    \int_E \limsup_{n \ra \infty} f_n \d{\mu} \not\geq \limsup_{n \ra \infty} \int_E f_n \d{\mu}.
\]
Consider \(\chi_{[n, \infty)}\). \(\text{LHS} = 0\), but \(\text{RHS} = \infty\).\footnote{뒤에서 하겠지만, \(\abs{f_n} \leq g\)인 \(g \in \mc{L}^{1}\) 가 있어야 합니다.}

\bigskip

\note{Step 4} If \(f\) is measurable, then \(f^+, f^- \geq 0\) are measurable. So for \(E \in \scr{F}\), define
\[
    \int_E f \d{\mu} = \int_E f^+ \d{\mu} - \int_E f^- \d{\mu},
\]
except for the case of \(\infty - \infty\).\footnote{둘 중 하나는 유한해야 한다.}

\defn. \note{Lebesgue Integrable} \(f\) is \textbf{Lebesgue integrable on \(E\) with respect to \(\mu\)} if \(f\) is measurable and
\[
    \int_E \abs{f} \d{\mu} = \int_E f^+ \d{\mu} + \int_E f^- \d{\mu} < \infty.
\]

다음 주에 함수들의 class를 볼텐데, 르벡 적분 가능한 함수를 다음과 같이 표기합니다.

\notation \(f\) is Lebesgue integrable \(\iff f \in \mc{L}^1(E, \mu)\). If \(\mu = m\), \(f \in \mc{L}^1(E)\).

Note that \(f \in \mc{L}^{1}(E, \mu) \iff f^+, f^- \in \mc{L}^{1}(E, \mu)\iff \abs{f} \in \mc{L}^{1}(E, \mu)\).

\pagebreak

\rmk
\begin{enumerate}
    \item \note{11.23} If \(f\) is measurable and bounded on \(E\) and \(\mu(E) < \infty\),
          \begin{center}
              \(\ds \int_E \abs{f} \d{\mu} \leq \int_E M \d{\mu} = M\mu(E) < \infty\) and \(f \in \mc{L}^{1}(E, \mu)\).
          \end{center}
    \item If \(f, g \in \mc{L}^{1}(E, \mu)\) and \(f \leq g\) on \(E\), then
          \begin{center}
              \(\chi_E (x) f^+(x) \leq \chi_E(x) g^+(x)\) and \(\chi_E(x) g^-(x) \leq \chi_E (x) f^-(x)\),
          \end{center}
          which implies
          \begin{center}
              \(\ds \int_E f^+ \d{\mu} \leq \int_E g^+ \d{\mu} < \infty\) and \(\ds \int_E g^- \d{\mu} \leq \int_E f^- \d{\mu} < \infty\).
          \end{center}
          Therefore monotonicity holds without the non-negative condition, i.e.
          \[
              \int_E f\d{\mu} \leq \int_E g \d{\mu}.
          \]
    \item If \(f \in \mc{L}^{1}(E, \mu)\) and \(c \in \R\), then \(cf \in \mc{L}^{1}(E, \mu)\) since
          \[
              \int_E \abs{c}\abs{f} \d{\mu} = \abs{c} \int_E \abs{f}\d{\mu} < \infty.
          \]
          If \(c < 0\), \((cf)^+ = -cf^-\), \((cf)^- = -cf^+\). Thus,
          \[
              \int_E cf \d{\mu} = \int_E (cf)^+ - \int_E (cf)^- \d{\mu} = -c \int_E f^- \d{\mu} - (-c) \int_E f^+ \d{\mu} = c\int_E f\d{\mu}.
          \]
    \item For measurable \(f\), if \(a\leq f(x) \leq b\) on \(E\) and \(\mu(E) < \infty\), (integrable since bounded)
          \[
              \int_E a \chi_E \d{\mu} \leq \int_E f\chi_E \d{\mu} \leq \int_E b \chi_E \d{\mu} \implies a \mu(E) \leq \int_E f \d{\mu} \leq b \mu(E).
          \]
    \item If \(f \in \mc{L}^{1}(E, \mu)\) and for \(A \in \scr{F}\) such that \(A \subset E\) then \(f \in \mc{L}^{1}(A, \mu)\) since
          \[
              \int_A \abs{f} \d{\mu} \leq \int_E \abs{f}\d{\mu} < \infty.
          \]
    \item Suppose that \(E\) is measurable and \(\mu(E) = 0\). If \(f\) is measurable, \(\min\{\abs{f}, n\}\chi_E\) is measurable and \(\min\{\abs{f}, n\}\chi_E \nearrow \abs{f}\chi_E\) as \(n \ra \infty\). By MCT,
          \[
              \int_E \abs{f} \d{\mu} = \lim_{n \ra \infty} \int_E \min\{\abs{f}, n\} \d{\mu} = 0
          \]
          since \(\ds \int_E \min\{\abs{f}, n\} \d{\mu} \leq \int_E n \d{\mu} = n \mu(E) = 0\).
          Thus \(f \in \mc{L}^{1}(E, \mu)\) and \(\ds \int_E f \d{\mu} = 0\).\footnote{Even if \(f \equiv \infty\). We defined \(0\cdot\infty = 0\).}
\end{enumerate}

\pagebreak

적분 입장에서 보면, measure가 0인 곳에서 적분을 하면, 의미가 없다고 생각할 수 있겠죠? 그러면 앞으로 그런걸 무시해도 된다고 해버리죠.

\defn. \note{Almost Everywhere} Let \(P = P(x)\) be a property.\footnote{Ex. \(f(x)\) is continuous.} We say that \(P\) holds \textbf{almost everywhere on \(E\) with respect to \(\mu\)} if
\begin{center}
    \(\exists N \in \scr{F}\) such that \(\mu(N) = 0\) and \(P\) holds for all \(x \in E \bs N\).
\end{center}

\notation We write \(p\) holds \(\mu\)-\ae on \(E\), and if \(E = X\), we omit `on \(E\)'.

\thm. \note{Markov Inequality} Let \(u \in \mc{L}^{1}(E, \mu)\). For all \(c > 0\),
\[
    \mu\paren{\{\abs{u} \geq c\} \cap E} \leq \frac{1}{c} \int_E \abs{u} \d{\mu}.
\]

\pf \(\ds \int_E \abs{u} \d{\mu} \geq \int_{E\cap \{\abs{u}\geq c\}} \abs{u} \d{\mu} \geq \int_{E\cap \{\abs{u}\geq c\}} c \d{\mu} = c \mu\paren{\{\abs{u} \geq c\} \cap E}\).

\thm. Let \(u\in \mc{L}^{1}(E, \mu)\). The following are equivalent.
\begin{enumerate}
    \item \(\ds \int_E \abs{u} \d{\mu} = 0\).
    \item \(u = 0\) \(\mu\)-\ae on \(E\).
    \item \(\mu\paren{\{x \in E : u(x) \neq 0\}} = 0\).
\end{enumerate}

\pf \\
\note{2\(\iff\)3} Clear since \(E\cap\{u\neq 0\} \in \scr{F}\).

\note{2\mimp1} \(\ds \int_E \abs{u} \d{\mu} = \int_{E \cap \{\abs{u} > 0\}} \abs{u} \d{\mu} + \int_{E \cap \{\abs{u} = 0\}} \abs{u} \d{\mu} = 0 + 0 = 0\).

\note{1\mimp3} By Markov inequality,
\[
    \mu\paren{\left\{\abs{u} \geq \frac{1}{n}\right\} \cap E} \leq n\int_E \abs{u} \d{\mu} = 0.
\]
Let \(n\ra \infty\), by continuity of measure, \(\mu\paren{\{\abs{u} > 0\} \cap E} = 0\).

\rmk Let \(A, B \in \scr{F}\). If \(B \subset A\) and \(\mu\paren{A \bs B} = 0\), then
\begin{center}
    \(\ds \int_A f \d{\mu} = \int_B f \d{\mu}\) for all \(f \in \mc{L}^{1}(A, \mu)\).
\end{center}

\pagebreak

\thm. If \(u \in \mc{L}^{1}(E, \mu)\) then \(u(x) \in \R\) \(\mu\)-\ae on \(E\).\footnote{\(u(x) = \infty\) 인 집합의 measure가 0이다.}

\pf \(\mu\paren{\{\abs{u} \geq 1\}\cap E} \leq \ds \int_E \abs{u} \d{\mu} < \infty\).\footnote{Continuity of measure를 사용하기 위해서는 첫 번째 집합의 measure가 유한해야 한다.} So,
\[
    \begin{aligned}
        \mu\paren{\{\abs{u} = \infty\} \cap E} & = \mu\paren{\bigcap_{n=1}^\infty \{x \in E : \abs{u(x)} \geq n\}}                                                            \\
                                               & = \lim_{n \ra \infty} \mu\paren{\{\abs{u} \geq n\} \cap E} \leq \limsup_{n\ra\infty} \frac{1}{n} \int_E \abs{u} \d{\mu} = 0.
    \end{aligned}
\]

\cor If \(u \in \mc{L}^{1}(E, \mu)\), then \(\ds \int_E u \d{\mu} = \int_{E \cap \{\abs{u} < \infty\}} u \d{\mu}\).

\thm. If \(f_1, f_2 \in \mc{L}^{1}(E, \mu)\), then \(f_1 + f_2 \in \mc{L}^{1}(E, \mu)\) and
\[
    \int_E \paren{f_1 + f_2} \d{\mu} = \int_E f_1 \d{\mu} + \int_E f_2 \d{\mu}.
\]

\pf Since \(\abs{f_1 + f_2} \leq \abs{f_1} + \abs{f_2}\), \(f_1+f_2 \in \mc{L}^{1}(E, \mu)\). Define \(f = f_1 + f_2\) and
\[
    N = \left\{x : \max\left\{f_1^+, f_1^-, f_2^+, f_2^-, f^+, f^-\right\} = \infty \right\}.
\]
Then by the above theorem, \(\mu(N) = 0\). So on \(E \bs N\),
\[
    f^+ - f^- = f_1^+ - f_1^- + f_2^+ - f_2^- \implies f^+ + f_1^- + f_2^- = f^- + f_1^+ + f_2^+.
\]
Then
\[
    \int_{E\bs N} f^+ \d{\mu} + \int_{E\bs N} f_1^- \d{\mu} + \int_{E\bs N} f_2^- \d{\mu} = \int_{E\bs N} f^-\d{\mu} + \int_{E\bs N} f_1^+\d{\mu} + \int_{E\bs N} f_2^+ \d{\mu}.
\]
Now using the fact that \(\mu(N) = 0\),
\[
    \int_{E \bs N} f \d{\mu} = \int_{E \bs N} f_1 \d{\mu} + \int_{E \bs N} f_2 \d{\mu} \implies \int_{E} f \d{\mu} = \int_{E} f_1 \d{\mu} + \int_{E} f_2 \d{\mu}.
\]
\pagebreak
