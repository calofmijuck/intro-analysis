%!TEX encoding = utf-8
\documentclass[12pt]{report}
\usepackage{kotex}
\usepackage{amsmath}
\usepackage{amsfonts}
\usepackage{amssymb}
\usepackage{amsthm}
\usepackage{mathtools}
\usepackage{geometry}
\usepackage{tikz}
\usepackage{pgfplots}
\geometry{
    top = 20mm,
    left = 20mm,
    right = 20mm,
    bottom = 20mm
}
\geometry{a4paper}

\pagenumbering{gobble}
\renewcommand{\baselinestretch}{1.4}
\newcommand{\numl}[1]{\item[\large\textbf{\sffamily #1.}]}
\newcommand{\num}[1]{\item[\textbf{\sffamily #1}]}

\newcommand{\ds}{\displaystyle}

\newcommand{\mf}[1]{\mathfrak{#1}}
\newcommand{\mc}[1]{\mathcal{#1}}
\newcommand{\bb}[1]{\mathbb{#1}}
\newcommand{\rmbf}[1]{\mathrm{\mathbf{#1}}}

\newcommand{\inv}{^{-1}}
\newcommand{\adj}{\text{*}}
\newcommand{\bs}{\setminus}
\renewcommand{\subset}{\subseteq}

\newcommand{\norm}[1]{\left\lVert #1 \right\rVert}
\newcommand{\abs}[1]{\left| #1 \right|}
\newcommand{\paren}[1]{\left( #1 \right)}
\newcommand{\seq}[1]{\left\{ #1 \right\}}
\renewcommand{\span}[1]{\left\langle #1 \right\rangle}

\newcommand{\ra}{\rightarrow}
\newcommand{\uc}{\overset{u}{\ra}}
\newcommand{\imp}{\implies}
\newcommand{\mimp}{\(\implies\)}
\newcommand{\mimpd}{\(\impliedby\)}
\newcommand{\miff}{\!\!\(\iff\)}
\newcommand{\mast}{\(\ast\)}

\newcommand{\R}{\mathbb{R}}
\newcommand{\N}{\mathbb{N}}
\newcommand{\Z}{\mathbb{Z}}
\newcommand{\Q}{\mathbb{Q}}
\newcommand{\C}{\mathbb{C}}

\newcommand{\inte}{\mathrm{int}}
\newcommand{\diam}{\text{diam}}
\newcommand{\dist}{\text{dist}}
\newcommand{\lint}[2]{\underline{\int_{#1}^{#2}}}
\newcommand{\uint}[2]{\overline{\int_{#1}^{#2}}}
\renewcommand{\d}[1]{\,d{#1}}

\let\oldexists\exists
\renewcommand{\exists}{\oldexists\,}

\begin{document}
\begin{center}
    \textbf{\Large 해석개론 및 연습 2 과제 \#8}\\
    \large 2017-18570 컴퓨터공학부 이성찬
\end{center}
\begin{enumerate}

    \numl{1} Set \(u_n = \abs{f - f_n}\), \(v_n = \abs{f - f_n}^{1/2}\). Then \(u_n\) is measurable, and \(v_n\) is also measurable since for \(a \in \R\),
    \[
        \{x \in \R : v_n > a\} = \{x \in \R : u_n > a^2\} \in \mf{M}.
    \]
    Next, since \(\lim_{n \ra \infty} \int_\R \abs{f - f_n} \d{x}= 0\) and \(\lim_{n \ra \infty} \int_\R \abs{f - f_n}^2 \d{x}= 0\), both \(\int_\R \abs{f - f_n} \d{x}\) and \(\int_\R \abs{f - f_n}^2 \d{x}\) should be finite for large enough \(n \in \N\). Therefore, there exists \(N \in \N\) such that \(n \geq N \implies u_n, v_n \in \mc{L}^{2}\). Now, from Schwarz inequality,
    \[
        0 \leq \int_\R \abs{f - f_n}^{3/2} \d{x} \leq \paren{\int_\R \abs{f - f_n}^2 \d{x}}^{1/2} \paren{\int_\R \abs{f - f_n} \d{x}}^{1/2}.
    \]
    Taking limit \(n \ra \infty\) on both sides will give \(\lim_{n \ra \infty} \int_\R \abs{f - f_n}^{3/2} \d{x} = 0\).

    \numl{2} Let \(\phi_n = e^{inx}\) for \(n \in \N\). We use the fact that \(\seq{\phi_n}_{n\in \N}\) is a complete orthonormal set. Let \(c_n = \frac{1}{n}\) and \(s_m = \sum_{n=1}^{m} c_n \phi_n\). Since \(\sum_{n=1}^{\infty}\abs{c_n}^2 = \sum_{n=1}^{\infty} \frac{1}{n^2} < \infty\), \(s_m\) converges to some function \(f \in \mc{L}^{2}[-\pi, \pi]\) by Riesz-Fischer theorem. Therefore \(\sum_{n=1}^{\infty} \frac{1}{n}e^{inx} \in \mc{L}^{2}[-\pi, \pi]\).

    Now suppose that \(g = \sum_{n=1}^{\infty} \frac{1}{\sqrt{n}} e^{inx} \in \mc{L}^{2}[-\pi, \pi]\). Then by {\sffamily Theorem 11.45}, we have that \(\sum_{n=1}^{\infty} \abs{\frac{1}{\sqrt{n}}}^2 = \int_{-\pi}^{\pi} \abs{g}^2 \d{x}\). However, the left hand side diverges to \(\infty\) (harmonic series), contradicting that \(g \in \mc{L}^{2}[-\pi, \pi]\) (the integral should be finite).

    \numl{3} First, boundedness can be checked easily by calculation.
    \[
        \norm{f_n}_2^2 = \int_{-\pi}^{\pi} \sin^2 nx \d{x} = \int_{-\pi}^{\pi} \frac{1 - \cos 2nx}{2} \d{x} = \pi. \quad (n \in \N)
    \]
    Next, to see that \(A = \{f_n : n \in \N\} \subset \mc{L}^{2}[-\pi, \pi]\) is closed, we show that all \(f_i \in A\) are isolated points. Direct calculation yields
    \[
        \begin{aligned}
            \norm{f_n - f_m}_2^2 & = \int_{-\pi}^{\pi} \abs{\sin nx - \sin mx}^2 \d{x}                                                                           \\
                                 & = \int_{-\pi}^{\pi}  \sin^2 nx \d{x} + \int_{-\pi}^{\pi} \sin^2 mx \d{x} - 2 \int_{-\pi}^{\pi}  \sin nx \sin mx \d{x} = 2\pi.
        \end{aligned}
    \]
    Thus, each \(f_i\) are \(\sqrt{2\pi}\) distance apart. So if we define
    \[
        B_{\frac{\sqrt{2\pi}}{2}}(f_i) = \left\{f \in \mc{L}^{2}[-\pi, \pi] : \norm{f - f_i}_2 < \frac{\sqrt{2\pi}}{2}\right\},
    \]
    then \(B_{\sqrt{2\pi}/2}(f_i) \bs \{f_i\} \cap A = \varnothing\). So \(A\) has no limit point, and \(A' = \varnothing\). Since \(A' \subset A\), \(A\) is closed.

    Finally, suppose that \(A\) is compact. Since \(A\) is infinite, \(A\) has a limit point in \(A\) by {\sffamily Theorem 2.37}. But \(A\) has no limit point, which is a contradiction. Thus \(A\) cannot be compact.

    \numl{4} Let \(\mu\) denote the Lebesgue measure for this problem. Let
    \[
        C_{m, N} = \bigcup_{n=N}^{\infty} \left\{x \in [a, b] : \abs{f_n(x) - f(x)} \geq \frac{1}{m} \right\}, \quad (m, N \in \N).
    \]
    Note that \(C_{m, N}\) is indeed measurable (because \(f_n, f\) are) and \(C_{m, N+1} \subset C_{m, N}\) for fixed \(m\). Also let \(C_m = \bigcap_{N=1}^{\infty} C_{m, N}\). For \(x \in [a, b]\), if \(f_n(x)\) converges to \(f(x)\), \(x\) cannot be in \(C_{m, N}\) for all \(N\), since for large enough \(N_0\), \(\abs{f_{N_0}(x) - f(x)} < \frac{1}{m}\) for fixed \(m\). So if \(C\) is the set of \(x \in [a, b]\) where \(f_n(x)\) does not converge to \(f(x)\), then \(\mu(C) = 0\) by the given condition. Also, we can write \(C = \bigcup_{m=1}^{\infty} C_m\). So using the completeness of Lebesgue measure, \(\mu(C_m) = 0\). (\(C_m \subset C\)) Since \(\mu([a, b]) < \infty\) and \(C_{m, N} \searrow C_m\), continuity of the measure from above gives \(\mu(C_{m, N}) \ra 0\) as \(N \ra \infty\).

    To construct a measurable set \(A \subset [a, b]\) with \(\mu(A) < \delta\), we can choose \(N_m \in \N\) for each \(m \in \N\) to construct a sequence \(\seq{N_m}_{m=1}^\infty\) such that \(\mu(C_{m, N_m}) < \frac{\delta}{2^m}\). If we set \(A = \bigcup_{m=1}^{\infty} C_{m, N_m}\), then
    \[
        \mu(A) \leq \sum_{m=1}^{\infty} \mu(C_{m, N_m}) < \sum_{m=1}^{\infty} \frac{\delta}{2^m} = \delta.
    \]
    Now we show that \(f_n(x)\) converges uniformly on \([a, b] \bs A\). We need to choose \(N\) large enough so that for \(n \geq N\), \(\abs{f_n(x) - f(x)} < \epsilon\) on \([a, b] \bs A\). For given \(\epsilon > 0\) choose \(m\) such that \(\frac{1}{m} < \epsilon\) and with this \(m\), choose \(N_\ast\) such that \(N_\ast > N_m\). Then if we consider
    \[
        \sup_{n \geq N_\ast,\; x \in [a, b] \bs A} \abs{f_n(x) - f(x)},
    \]
    we see that all \(x \in [a, b]\) such that \(\abs{f_n(x) - f(x)} \geq \frac{1}{m}\) for \(n \geq N_\ast\), \(m \in \N\) has been taken out by \(A\).\footnotemark[1] Therefore,
    \[
        \sup_{n \geq N_\ast,\; x \in [a, b] \bs A} \abs{f_n(x) - f(x)} < \frac{1}{m} < \epsilon.
    \]
    \footnotetext[1]{By the choice of \(A = \bigcup_{m=1}^{\infty} C_{m, N_m}\), for each \(m\), all [\(x \in [a, b]\) such that \(\abs{f_n(x) - f(x)} \geq \frac{1}{m}\) for \(n \geq N_m\)] has been taken out. But by the choice of \(N_m\), there may still be \(x \in [a, b] \bs A\) such that \(n < N_m\) and \(\abs{f_n(x) - f(x)} \geq \frac{1}{m}\). Thus, we have to set \(N_m\) large enough so that \(\abs{f_n(x) - f(x)} < \frac{1}{m}\) for \(n \geq N_m\).}

    \numl{5} First we prove the following claim.

    \quad {\sffamily \bfseries Claim.} Let \(A\) be a measurable subset of \([-\pi, \pi]\). Then
    \begin{center}
        \(\ds \int_A \sin nx \d{x}, \int_A \cos nx \d{x} \ra 0\) as \(n \ra \infty\).
    \end{center}

    \quad {\sffamily Proof.} On \(\mc{L}^{2}[-\pi, \pi]\), we consider the Fourier coefficients of \(\chi_A\). Then by Bessel's inequality,
    \[
        c_n = \frac{1}{2\pi} \int_{-\pi}^{\pi} \chi_A e^{inx} \d{x} = \frac{1}{2\pi} \int_A e^{inx} \d{x} \ra 0
    \]
    as \(n \ra \infty\). Therefore \(\int_A \sin nx \d{x}, \int_A \cos nx \d{x} \ra 0\).

    For measurable \(A \subset E\), we have \(\int_A \sin n_kx \d{x} \ra 0\) and
    \[
        2 \int_A (\sin n_k x)^2 \d{x} = \int_A (1 - \cos 2n_k x) \d{x} = m(A) - \int_A \cos 2n_k x\d{x} \ra m(A)
    \]
    as \(k \ra \infty\) by the above claim and the half-angle formula.

    Now consider \(f(x) = \lim_{k \ra \infty} \sin n_k x\) for \(x \in E\). \(f\) is measurable, and since \(\abs{2\sin n_k x} \leq 2\) for all \(x \in E\), we can use LDCT and conclude that
    \[
        m(A) = \lim_{k \ra \infty} \int_A 2\sin^2 n_k x\d{x} = \int_A 2f^2\d{x}.
    \]
    Therefore \(\int_A \left[2f^2 - 1\right] \d{x} = 0\) for all measurable \(A \subset E\). By {\sffamily Problem 7} of {\sffamily Homework 6}, \(f^2(x) = \frac{1}{2}\) a.e. on \(E\). Now consider \(f\inv(1/\sqrt{2})\) and \(f\inv(-1/\sqrt{2})\). \(f\inv(1/\sqrt{2}) \subset E\) trivially, and
    \[
        0 = \int_{f\inv(1/\sqrt{2})} f(x) \d{x} =\int_{f\inv(1/\sqrt{2})} \frac{1}{\sqrt{2}} \d{x} = \frac{1}{\sqrt{2}} m(f\inv(1/\sqrt{2})).
    \]
    Thus \(m(f\inv(1/\sqrt{2})) = 0\), and similarly \(m(f\inv(-1/\sqrt{2})) = 0\). Therefore, \[m(E) = m(f\inv(1/\sqrt{2})) + m(f\inv(-1/\sqrt{2})) = 0.\]

    \numl{6} Suppose that \(\sin nx \geq \delta\) for all \(x \in E\) holds for infinitely many \(n \in \N\). Then from
    \[
        \int_E \sin nx \d{x} \geq \int_E \delta \d{x} = \delta m(E) > 0.
    \]
    But the left hand side should approach 0 as \(n \ra \infty\). (\(\because\;\){\sffamily Claim} in {\sffamily Problem 5}) This is impossible, so \(\sin nx \geq \delta\) can only hold for finitely many \(n \in \N\).

    \numl{7} If \(f \sim 0\) or \(g \sim 0\) in \(\mc{L}^{2}(\mu)\), the statement is trivial, so we assume that this is not the case. (In fact, the statement is false when \(f \sim 0\) but \(g \not\sim 0\).)

    (\mimpd) Suppose that \(\exists c \in \C\) such that \(g(x) = cf(x)\) \(\mu\)-a.e.. Then \(f \overline{g} = \overline{c} \abs{f}^2\), \(\abs{g}^2 = \abs{c}^2 \abs{f}^2\) \(\mu\)-a.e.. Therefore,
    \[
        \text{LHS} = \abs{\int f\overline{g} \d{\mu}}^2 =\abs{\int \overline{c} \abs{f}^2 \d{\mu}}^2 = \abs{c}^2 \norm{f}_2^4,
    \]
    \[
        \text{RHS} = \paren{\int \abs{f}^2 \d{\mu}} \paren{\int \abs{c}^2 \abs{f}^2 \d{\mu}} = \abs{c}^2 \norm{f}_2^4.
    \]
    (\mimp) Denote \(\span{f, g} = \int f\overline{g} \d{\mu}\). The assumption can be written as
    \[
        \abs{\span{f, g}}^2 = \norm{f}_2^2 \norm{g}_2^2.
    \]
    Let \(u = f - \frac{\span{f, g}}{\norm{g}_2^2}g\). Then \(u\) is indeed measurable. Now by calculation,
    \[
        \begin{aligned}
            \int \abs{u}^2 \d{\mu} & = \int \paren{f - \frac{\span{f, g}}{\norm{g}_2^2}g} \overline{\paren{f - \frac{\span{f, g}}{\norm{g}_2^2}g}} \d{\mu}                                                                                                            \\
                                   & = \int \abs{f}^2 \d{\mu} - \int \frac{\span{f, g}}{\norm{g}_2^2} \overline{f} g \d{\mu} - \int \frac{\overline{\span{f, g}}}{\norm{g}_2^2}f\overline{g}\d{\mu} + \int \frac{\abs{\span{f, g}}^2}{\norm{g}_2^4} \abs{g}^2 \d{\mu} \\
                                   & = \norm{f}_2^2 - \frac{\span{f, g}}{\norm{g}_2^2} \span{g, f} - \frac{\overline{\span{f, g}}}{\norm{g}_2^2}\span{f, g} + \frac{\abs{\span{f, g}}^2}{\norm{g}_2^4} \cdot\norm{g}_2^2                                              \\
                                   & =\norm{f}_2^2 - \frac{\span{f, g}}{\norm{g}_2^2} \overline{\span{f, g}} - \frac{\overline{\span{f, g}}}{\norm{g}_2^2}\span{f, g} + \norm{f}_2^2                                                                                  \\
                                   & = \norm{f}_2^2 - \norm{f}_2^2 - \norm{f}_2^2 + \norm{f}_2^2 = 0.
        \end{aligned}
    \]
    \(\span{f, g} = \overline{\span{g, f}}\) and the given assumption was used. Since \(\abs{u}^2 \geq 0\), \(u = 0\) \(\mu\)-a.e. by {\sffamily Problem 5} of {\sffamily Homework 6}. Therefore, \(g = cf\) for some constant \(c \in \C\) \(\mu\)-a.e..
\end{enumerate}
\end{document}
