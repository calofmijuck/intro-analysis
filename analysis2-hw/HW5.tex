%!TEX encoding = utf-8
\documentclass[12pt]{report}
\usepackage{kotex}
\usepackage{amsmath}
\usepackage{amsfonts}
\usepackage{amssymb}
\usepackage{amsthm}
\usepackage{mathtools}
\usepackage{geometry}
\geometry{
    top = 20mm,
    left = 20mm,
    right = 20mm,
    bottom = 20mm
}
\geometry{a4paper}

\pagenumbering{gobble}
\renewcommand{\baselinestretch}{1.4}
\newcommand{\numl}[1]{\item[\large\textbf{\sffamily #1.}]}
\newcommand{\num}[1]{\item[\textbf{\sffamily #1}]}

\newcommand{\ds}{\displaystyle}

\newcommand{\mf}[1]{\mathfrak{#1}}
\newcommand{\mc}[1]{\mathcal{#1}}
\newcommand{\bb}[1]{\mathbb{#1}}
\newcommand{\rmbf}[1]{\mathrm{\mathbf{#1}}}

\newcommand{\inv}{^{-1}}
\newcommand{\adj}{\text{*}}
\newcommand{\bs}{\setminus}
\renewcommand{\subset}{\subseteq}

\newcommand{\norm}[1]{\left\lVert #1 \right\rVert}
\newcommand{\abs}[1]{\left| #1 \right|}
\newcommand{\paren}[1]{\left( #1 \right)}
\newcommand{\seq}[1]{\left\{ #1 \right\}}
\renewcommand{\span}[1]{\left\langle #1 \right\rangle}

\newcommand{\ra}{\rightarrow}
\newcommand{\uc}{\overset{u}{\ra}}
\newcommand{\imp}{\implies}
\newcommand{\mimp}{\(\implies\)}
\newcommand{\mimpd}{\(\impliedby\)}
\newcommand{\miff}{\!\!\(\iff\)}
\newcommand{\mast}{\(\ast\)}

\newcommand{\R}{\mathbb{R}}
\newcommand{\N}{\mathbb{N}}
\newcommand{\Z}{\mathbb{Z}}
\newcommand{\Q}{\mathbb{Q}}
\newcommand{\C}{\mathbb{C}}

\newcommand{\inte}{\mathrm{int}}
\newcommand{\diam}{\text{diam}}
\newcommand{\dist}{\text{dist}}
\newcommand{\lint}[2]{\underline{\int_{#1}^{#2}}}
\newcommand{\uint}[2]{\overline{\int_{#1}^{#2}}}
\renewcommand{\d}[1]{\,d{#1}}

\let\oldexists\exists
\renewcommand{\exists}{\oldexists\,}

\begin{document}
\begin{center}
    \textbf{\Large 해석개론 및 연습 2 과제 \#5}\\
    \large 2017-18570 컴퓨터공학부 이성찬
\end{center}
\begin{enumerate}

    \numl{1} {\sffamily Show that \(\Sigma \subset \mc{P}(S)\) is an algebra on \(S\) if and only if \(\Sigma \subset \mc{P}(S)\) is a ring on \(S\) with \(S \in \Sigma\).}

    (\mimp) \(S\in \Sigma\) is trivial. If \(A, B\in \Sigma\), then \(A \cup B\in \Sigma\). (algebra) Also, it follows that
    \[
        A \bs B = S \bs \big((S \bs A) \cup B \big) \in \Sigma,
    \]
    since \(S \bs A \in \Sigma\) and \((S \bs A) \cup B \in \Sigma\). Therefore \(\Sigma\) is a ring on \(S\) with \(S \in \Sigma\).

    (\mimpd) \(S\in \Sigma\) is trivial. If \(A, B\in \Sigma\), then \(A \cup B\in \Sigma\). (ring) Since \(S, A\in \Sigma\), \(S \bs A \in \Sigma\). (ring) Therefore \(\Sigma\) is an algebra on \(S\).

    \vspace*{10px}

    {\sffamily Show that \(\Sigma \hspace*{-0.6px} \subset \hspace*{-0.6px} \mc{P}(S)\) is a \(\sigma\)-algebra on \(S\) if and only if \(\Sigma \hspace*{-0.5px} \subset \hspace*{-0.5px} \mc{P}(S)\) is a \(\sigma\)-ring on \(S\) with \(S \in \Sigma\).}

    By the proof above, we only need to show that any countable union of elements of \(\Sigma\) is also an element of \(\Sigma\). Let \(A_n \in \Sigma\) (\(n = 1, 2, \dots\)). Then \(\ds \bigcup_{n=1}^\infty A_n \in \Sigma\) holds for both cases where \(\Sigma\) is a \(\sigma\)-algebra or a \(\sigma\)-ring. Therefore the statement is proven.

    \numl{2}
    \begin{enumerate}
        \item[(i)] \(E = f\inv(S) \in f\inv(\Sigma)\), since \(S \in \Sigma\). (algebra)
        \item[(ii)] Take two elements \(f\inv(A), f\inv(B) \in f\inv(\Sigma)\), where \(A, B \in \Sigma\). Then
            \[
                f\inv(A) \cup f\inv(B) = f\inv(A \cup B) \in f\inv(\Sigma)
            \]
            since \(A \cup B \in \Sigma\). (algebra)
        \item[(iii)] Take an element \(f\inv(A) \in f\inv(\Sigma)\), where \(A \in \Sigma\). Then
            \[
                f\inv(S) \bs f\inv(A) \overset{(\ast)}{=} f\inv\paren{S \bs A} \in f\inv(\Sigma)
            \]
            since \(S \bs A \in \Sigma\). (algebra)
        \item[(iv)] For \(f\inv(A_n) \in f\inv(\Sigma)\) where \(A_n \in \Sigma\), (\(n = 1, 2, \dots\)) the following holds.
            \[
                \bigcup_{n=1}^\infty f\inv(A_n) = f\inv\paren{\bigcup_{n=1}^\infty A_n} \in f\inv(\Sigma)
            \]
            since \(\bigcup_{n=1}^\infty A_n \in \Sigma\). (\(\sigma\)-algebra)
    \end{enumerate}

    Therefore, \(f\inv(\Sigma)\) is a \(\sigma\)-algebra on \(E\).

        {\sffamily \bfseries Proof of (\mast)}
    \[
        \begin{aligned}
            x \in f\inv(S) \bs f\inv(A) \iff & x \in f\inv(S) \land x \notin f\inv(A)           \\
            \iff                             & f(x) \in S \land f(x) \notin A                   \\
            \iff                             & f(x) \in S\bs A \iff x \in f\inv\paren{S \bs A}.
        \end{aligned}
    \]

    \numl{3} Let \(\Sigma_n\) (\(n = 1, 2, \dots\)) be \(\sigma\)-algebras on \(S\), and define \(\Sigma = \bigcap_{n} \Sigma_n\). It is trivial that \(S \in \Sigma\).

    If \(A, B \in \Sigma\), then \(A, B \in \Sigma_i\) for all \(i \in \N\). Then \(S \bs A \in \Sigma_i\) and \(A \cup B \in \Sigma_i\) for all \(i \in \N\). Therefore \(S\bs A \in \Sigma\) and \(A \cup B \in \Sigma\).

    Finally, if \(A_j \in \Sigma\) for \(j = 1, 2, \dots\), then \(A_j \in \Sigma_i\) for all \(i, j \in \N\). So \(\bigcup_{j=1}^\infty A_j \in \Sigma_i\) for all \(i \in \N\), since \(\Sigma_i\) are \(\sigma\)-algebras. Therefore \(\bigcup_{j=1}^\infty A_j \in \Sigma\).

    Thus an arbitrary intersection of \(\sigma\)-algebras on \(S\) is a \(\sigma\)-algebra on \(S\). For the case of unions, consider
    \[
        \Sigma_1 = \{\varnothing, \{1\}, \{2, 3\}, \{1, 2, 3\}\}, \quad \Sigma_2 = \{\varnothing, \{2\}, \{1, 3\}, \{1, 2, 3\}\}.
    \]
    Then \(\{1\}, \{2\} \in \Sigma_1 \cup \Sigma_2\), but \(\{1, 2\}\notin \Sigma_1\cup \Sigma_2\).

    \numl{4} We first check that \(\Sigma\) is a \(\sigma\)-ring. For \(A, B \in \Sigma\), it is easy to see that \(A\cup B, A\bs B\) are both in \(\Sigma\). Also if \(A_i \in \Sigma\) (\(i = 1, 2, \dots\)), then \(\bigcup_{i=1}^\infty A_i \subset S \in \mc{P(S)} = \Sigma\).

    Now let \(A_i \in \Sigma\) be pairwise disjoint sets, and let \(A = \bigcup_{i=1}^\infty A_i\).
    \begin{enumerate}
        \item[(1)] Suppose \(x \in A\). Then because \(A_i\) are disjoint, \(\exists N \in \N\) such that \(x \in A_N\). So we can see that \(\mu_1(A_i) = 1\) if \(i = N\) and 0 otherwise. If \(x \notin A\), \(\mu_1(A_i) = 0\) for all \(i\in \N\). Therefore, for both cases,
            \[
                \mu_1\paren{\bigcup_{i=1}^\infty A_i} = \begin{cases}
                    1 = \sum_{i=1}^\infty \mu_1(A_i) & (x \in A)    \\
                    0 = \sum_{i=1}^\infty \mu_1(A_i) & (x \notin A)
                \end{cases}
            \]
            and \(\mu_1\) is a measure on \(\Sigma\).
        \item[(2)] If \(\abs{A_N} = \infty\) for some \(N \in \N\), then \(\mu_2\paren{\bigcup_{i=1}^\infty A_i} = \sum_{i=1}^\infty \mu_2(A_i) = \infty\). Now suppose that \(\abs{A_i} < \infty\) for all \(i \in \N\). If \(\sum_{i=1}^\infty \mu_2(A_i) = \sum_{i=1}^\infty \abs{A_i} < \infty\), then
            \[
                \mu_2\paren{\bigcup_{i=1}^\infty A_i} = \abs{\bigcup_{i=1}^\infty A_i} = \sum_{i=1}^\infty \abs{A_i} < \infty.
            \]
            If \(\sum_{i=1}^\infty \mu_2(A_i) = \infty\), we can take any \(K > 0\) and find \(M \in \N\) such that \(\sum_{i=1}^M \mu_2(A_i) > K\). Then
            \[
                \mu_2\paren{\bigcup_{i=1}^\infty A_i} \geq \mu_2\paren{\bigcup_{i=1}^M A_i} = \sum_{i=1}^M \mu_2(A_i) > K
            \]
            for all \(K > 0\). Thus \(\mu_2(\bigcup_{i=1}^\infty A_i) = \sum_{i=1}^\infty \mu_2(A_i) = \infty\). Overall, \(\mu_2\) is a measure on \(\Sigma\).
    \end{enumerate}

    \numl{5} Define \(B_n = A_n \bs A_{n+1}\) for \(n = 1, 2, \dots\). Then we directly see that \(B_n\) are pairwise disjoint. Note that
    \[ \tag{\mast}
        \sum_{i=1}^n \mu(B_i) = \sum_{i=1}^n \left[\mu(A_i) - \mu(A_{i+1})\right] = \mu(A_1) - \mu(A_{n+1}).
    \]
    Therefore,
    \[
        \begin{aligned}
            \mu\paren{\bigcap_{i=1}^\infty A_i} & = \mu\paren{A_1 \bs \bigcup_{i=1}^\infty B_i} = \mu(A_1) - \mu\paren{\bigcup_{i=1}^\infty B_i} = \mu(A_1) - \sum_{i=1}^\infty \mu(B_i) \\
                                                & =\lim_{n\ra\infty} \left[\mu(A_1) - \sum_{i=1}^n \mu(B_i)\right] = \lim_{n\ra \infty} \mu(A_{n+1}) = \lim_{n\ra \infty} \mu(A_{n})
        \end{aligned}
    \]
    by (\mast).

    \numl{6} Define \(B_1 = A_1\), \(B_n = A_n \bs \bigcup_{i=1}^{n-1} A_i\) for \(n \geq 2\). Then \(B_n\) are pairwise disjoint. Since \(B_n \subset A_n\) for all \(n \in \N\), we have that \(\mu(B_n) \leq \mu(A_n)\). Also note that
    \[
        \bigcup_{i=1}^\infty B_i = \bigcup_{i=1}^\infty A_i \in \mc{F}
    \]
    since \(\mc{F}\) is a \(\sigma\)-algebra. Then
    \[
        \mu\paren{\bigcup_{i=1}^\infty A_i} = \mu\paren{\bigcup_{i=1}^\infty B_i} = \sum_{i=1}^\infty \mu(B_i) \leq \sum_{i=1}^\infty \mu(A_i).
    \]
\end{enumerate}
\end{document}
