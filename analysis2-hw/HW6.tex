%!TEX encoding = utf-8
\documentclass[12pt]{report}
\usepackage{kotex}
\usepackage{amsmath}
\usepackage{amsfonts}
\usepackage{amssymb}
\usepackage{amsthm}
\usepackage{mathtools}
\usepackage{geometry}
\geometry{
    top = 20mm,
    left = 20mm,
    right = 20mm,
    bottom = 20mm
}
\geometry{a4paper}

\pagenumbering{gobble}
\renewcommand{\baselinestretch}{1.4}
\newcommand{\numl}[1]{\item[\large\textbf{\sffamily #1.}]}
\newcommand{\num}[1]{\item[\textbf{\sffamily #1}]}

\newcommand{\ds}{\displaystyle}

\newcommand{\mf}[1]{\mathfrak{#1}}
\newcommand{\mc}[1]{\mathcal{#1}}
\newcommand{\bb}[1]{\mathbb{#1}}
\newcommand{\rmbf}[1]{\mathrm{\mathbf{#1}}}

\newcommand{\inv}{^{-1}}
\newcommand{\adj}{\text{*}}
\newcommand{\bs}{\setminus}
\renewcommand{\subset}{\subseteq}

\newcommand{\norm}[1]{\left\lVert #1 \right\rVert}
\newcommand{\abs}[1]{\left| #1 \right|}
\newcommand{\paren}[1]{\left( #1 \right)}
\newcommand{\seq}[1]{\left\{ #1 \right\}}
\renewcommand{\span}[1]{\left\langle #1 \right\rangle}

\newcommand{\ra}{\rightarrow}
\newcommand{\uc}{\overset{u}{\ra}}
\newcommand{\imp}{\implies}
\newcommand{\mimp}{\(\implies\)}
\newcommand{\mimpd}{\(\impliedby\)}
\newcommand{\miff}{\!\!\(\iff\)}
\newcommand{\mast}{\(\ast\)}

\newcommand{\R}{\mathbb{R}}
\newcommand{\N}{\mathbb{N}}
\newcommand{\Z}{\mathbb{Z}}
\newcommand{\Q}{\mathbb{Q}}
\newcommand{\C}{\mathbb{C}}

\newcommand{\inte}{\mathrm{int}}
\newcommand{\diam}{\text{diam}}
\newcommand{\dist}{\text{dist}}
\newcommand{\lint}[2]{\underline{\int_{#1}^{#2}}}
\newcommand{\uint}[2]{\overline{\int_{#1}^{#2}}}
\renewcommand{\d}[1]{\,d{#1}}

\let\oldexists\exists
\renewcommand{\exists}{\oldexists\,}

\begin{document}
\begin{center}
    \textbf{\Large 해석개론 및 연습 2 과제 \#6}\\
    \large 2017-18570 컴퓨터공학부 이성찬
\end{center}
\begin{enumerate}

    \numl{1} Let \(A = I_1 \times I_2 \times \cdots \times I_p \subset \R^p\) where \(I_i \subset \R\) and \(I_i\) has endpoints \(a_i, b_i \in \R\), \(a_i \leq b_i\) for \(i = 1, 2, \dots, p\). \(I_i\) can be any type of intervals - \([a, b], (a, b), (a, b], [a, b)\). Now let \(\epsilon > 0\) be given.

    Suppose that \(m(A) < \epsilon\). We can take \(F = \varnothing\), and then \(F\) is closed and satisfies \(m(A) \leq m(F) + \epsilon\). Now assume that \(m(A) \geq \epsilon\). Consider a function \(f: \R \ra \R\) defined as
    \[
        f(x) = m\paren{\prod_{i=1}^p [a_i + x, b_i - x]} = \prod_{i=1}^p (b_i - a_i - 2x),
    \]
    where \(0 \leq x \leq M = \min_{1 \leq i \leq p} \frac{\abs{b_i - a_i}}{2}\). Note that \(a_i \neq b_i\) for all \(i\) since \(m(A) \geq \epsilon \neq 0\). It is trivial that \(f\) is continuous and decreasing. Since \(f(0) = m(A), f(M) = 0\), there exists \(c \in (0, M)\) such that \(f(c) = m(A) - \epsilon\). (Intermediate Value Theorem) With this \(c\), construct a closed box \(F = \prod_{i=1}^{p} [a_i + c, b_i - c]\). Then \(F \subset A\) and \(m(A) \leq m(F) + \epsilon\) holds.

    To find an open set \(G\), consider a function \(g: \R \ra \R\) defined as
    \[
        g(x) = m\paren{\prod_{i=1}^p (a_i-x, b_i+x)} = \prod_{i=1}^p (b_i - a_i + 2x),
    \]
    where \(x \geq 0\). In a similar manner, \(g\) is continuous and increasing, diverges to \(\infty\) as \(x \ra \infty\). Thus there exists \(d \in (0, \infty)\) such that \(g(d) = m(A) + \epsilon\). With this \(d\), construct an open box \(G = \prod_{i=1}^p (a_i - d, b_i + d)\). Then \(A \subset G\) and \(m(G) - \epsilon \leq m(A)\) holds.

    Since any elementary set is a finite union of disjoint intervals, take open sets/closed sets for each interval as above, and union open sets/closed sets respectively. Since \(m\) is additive on \(\Sigma\), it can be concluded that \(m\) is regular on \(\Sigma\).

    \numl{2} It is enough to show the following claim.

        {\sffamily \bfseries Claim.} \(f\inv\bigl((a, \infty]\bigr) = \{x : f(x) > a\}\) is an interval.

        {\sffamily Proof.} Suppose that \(t \in f\inv\bigl((a, \infty]\bigr)\). Then for all \(u \geq t\), \(a < f(t) \leq f(u) < \infty\). So \(u \in f\inv\bigl((a, \infty]\bigr)\). Thus this interval is one of \((-\infty, \infty)\), \((z, \infty)\), \([z, \infty)\). (\(z\) is some constant)

    \numl{3} Define \(N = \{x \in \R : f(x) \neq g(x)\}\). Then \(m(N) = 0\), and by the completeness of the Lebesgue measure, any subset of \(N\) has measure \(0\). Now we show that \(g\) is measurable. Take any \(a \in \R\).
    \[
        \begin{aligned}
            \{x \in \R : g(x) > a\} & = \{x \in \R \bs N : g(x) > a\} \cup \{x \in N : g(x) > a\}  \\
                                    & =\{x \in \R \bs N : f(x)  > a\} \cup \{x \in \N : g(x) > a\}
        \end{aligned}
    \]
    Since \(\{x \in \N : g(x) > a\} \subset N\), the set \(\{x \in \N : g(x) > a\}\) is measurable and has measure zero. Also, \(\{x \in \R \bs N : f(x) > a\}\) is measurable. Thus \(\{x \in \R : g(x) > a\}\) is measurable for all \(a \in \R\).

    \numl{4} \(f_n(x)\) converges \miff \(\forall M > 0\), \(\exists N \in \N\) such that \(m, n \geq N \implies \abs{f_n(x) - f_m(x)} < \frac{1}{M}\).
    The set \(C\) which \(f_n(x)\) converges can be written as
    \[
        C = \bigcap_{M=1}^{\infty} \bigcup_{N=1}^\infty \bigcap_{n=N}^{\infty} \bigcap_{m=N}^{\infty} \left\{x : \abs{f_n - f_m} < \frac{1}{M}\right\}.
    \]
    We take intersections for \(M, n, m\) since \(\abs{f_n - f_m} < \frac{1}{M}\) has to hold for all values of \(M, n, m\). However, we take unions for \(N\) since we have to include all \(N\) if such \(N\) can make \(\abs{f_n - f_m} < \frac{1}{M}\) hold.

    Since \(f_n\) are measurable, \(f_n - f_m\) is measurable and \(\abs{f_n - f_m}\) is also measurable. Thus the set \(\left\{x : \abs{f_n - f_m} < \frac{1}{M}\right\}\) is measurable, and its countable union \(C\) is measurable.

    \numl{5} Define \(E_n =\{x \in E : f(x) > \frac{1}{n}\}\) for \(n \in \N\), and let \(A = \bigcup_{n=1}^{\infty} E_n\). We show that \(\mu(A) = 0\), by showing that \(\mu(E_n) = 0\) for all \(n \in \N\). Suppose that \(\mu(E_n) > 0\) for some \(n \in \N\). Then
    \[
        \int_E f \d{\mu} = \int_{E\bs E_n} f \d{\mu} + \int_{E_n} f \d{\mu} \geq \int_{E_n} f \d{\mu} \geq \int_{E_n} \frac{1}{n} \d{\mu} = \frac{\mu(E_n)}{n} > 0,
    \]
    leading to a contradiction. Thus \(\mu(E_n) = 0\) for all \(n \in \N\), which leads to \(\mu(A) = 0\), since \(0 \leq \mu(A) \leq \sum_{n=1}^{\infty} \mu(E_n) = 0\). Therefore \(\{x : f(x) > 0\}\) is a measure zero set.

    \numl{6} Since \(\int_0^1 g(1-x) \d{x} = \int_0^1 g(x)\d{x}\), \(\int_0^1 f_n(x) \d{x} = \int_0^1 g(x) \d{x} = \frac{1}{2}\) regardless of the parity of \(n\). Check that
    \[
        \{f_n(x)\} = \begin{cases}
            1, 0, 1, 0, 1, \dots & (x \in [0, 1/2]) \\
            0, 1, 0, 1, 0, \dots & (x \in (1/2, 1])
        \end{cases}
    \]
    Therefore for \(0\leq x\leq 1\), \(\inf_{k\geq n} f_k (x) = 0\) for any \(n\in \N\). So we can conclude that \(\liminf_{n \ra \infty} f_n(x) = 0\).

    \numl{7} We show that \(N = \{x \in E : f(x) \neq 0\}\) has measure zero. Let \(S_1 = \{x \in E: f(x) > 0\}\), then \(S_1\) is measurable. So \(\int_{S_1} f \d{\mu} = 0\). Since \(f\geq 0\) on \(S_1\), we use {\sffamily Problem 5} and conclude that \(\mu(S_1) = 0\). Similarly, define \(S_2 = \{x \in E : f(x) < 0\}\). Then \(S_2\) is measurable, so \(\int_{S_2} f \d{\mu} = 0\). Since \(-f \geq 0\) on \(S_2\), we use {\sffamily Probelm 5} again and conclude that \(\mu(S_2) = 0\). Therefore, \(\mu(N) = \mu(S_1) + \mu(S_2) = 0\), and \(f(x) = 0\) \(\mu\)-almost everywhere on \(E\).

    \pagebreak

    \numl{8} {\sffamily (\(\implies\))} For any \(A \in \mf{M}\), set \(N_A = \{x \in A : f(x) \neq g(x)\}\). Since \(\mu(N_A) = 0\), \(\int_{N_A} f \d{\mu} = \int_{N_A} g \d{\mu} = 0\). Now we have
    \[
        \int_A f\d{\mu} = \int_{A \bs N_A} f \d{\mu} + \int_{N_A} f \d{\mu} = \int_{A \bs N_A} g \d{\mu} + \int_{N_A} g \d{\mu} = \int_A g\d{\mu}.
    \]
    {\sffamily (\(\impliedby\))} Since \(\int_A f \d{\mu}, \int_B f \d{\mu}\) are finite, (\(f, g \in \mc{L}^{1}(X, \mu)\)) we have \(\int_{A} (f-g) \d{\mu} = 0\) for any \(A \in \mf{M}\). By {\sffamily Problem 7}, \(f - g = 0\) \(\mu\)-almost everywhere since \(A\) is any measurable subset of \(X\). Therefore \(f = g\) \(\mu\)-almost everywhere.
\end{enumerate}
\end{document}
