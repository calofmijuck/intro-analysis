%!TEX encoding = utf-8
\documentclass[12pt]{report}
\usepackage{kotex}
\usepackage{amsmath}
\usepackage{amsfonts}
\usepackage{amssymb}
\usepackage{mathtools}
\usepackage{geometry}
\geometry{
    top = 20mm,
    left = 20mm,
    right = 20mm,
    bottom = 20mm
}
\geometry{a4paper}

\pagenumbering{gobble}
\renewcommand{\baselinestretch}{1.4}
\newcommand{\numl}[1]{\item[\large\textbf{\sffamily #1.}]}
\newcommand{\num}[1]{\item[\textbf{\sffamily #1}]}

\newcommand{\ds}{\displaystyle}

\newcommand{\mf}[1]{\mathfrak{#1}}
\newcommand{\mc}[1]{\mathcal{#1}}
\newcommand{\bb}[1]{\mathbb{#1}}
\newcommand{\rmbf}[1]{\mathrm{\mathbf{#1}}}

\newcommand{\inv}{^{-1}}
\newcommand{\adj}{\text{*}}
\newcommand{\bs}{\setminus}
\renewcommand{\subset}{\subseteq}

\newcommand{\norm}[1]{\left\lVert #1 \right\rVert}
\newcommand{\abs}[1]{\left| #1 \right|}
\newcommand{\paren}[1]{\left( #1 \right)}
\newcommand{\seq}[1]{\left\{ #1 \right\}}
\renewcommand{\span}[1]{\left\langle #1 \right\rangle}

\newcommand{\ra}{\rightarrow}
\newcommand{\uc}{\overset{u}{\ra}}
\newcommand{\imp}{\implies}
\newcommand{\mimp}{\(\implies\)}
\newcommand{\mimpd}{\(\impliedby\)}
\newcommand{\miff}{\!\!\(\iff\)}

\newcommand{\R}{\mathbb{R}}
\newcommand{\N}{\mathbb{N}}
\newcommand{\Z}{\mathbb{Z}}
\newcommand{\Q}{\mathbb{Q}}
\newcommand{\C}{\mathbb{C}}

\newcommand{\inte}{\mathrm{int}}
\newcommand{\diam}{\text{diam}}
\newcommand{\dist}{\text{dist}}
\newcommand{\lint}[2]{\underline{\int_{#1}^{#2}}}
\newcommand{\uint}[2]{\overline{\int_{#1}^{#2}}}
\renewcommand{\d}[1]{\,d{#1}}

\let\oldexists\exists
\renewcommand{\exists}{\oldexists\,}

\begin{document}
\begin{center}
    \textbf{\Large 해석개론 및 연습 2 과제 \#1}\\
    \large 2017-18570 컴퓨터공학부 이성찬
\end{center}
\begin{enumerate}
    \numl{1} Suppose that \(\seq{f_n}_{n=1}^\infty\) converges uniformly to \(f\) on \(X\), and that \(\abs{f_n} \leq M_n\) for all \(n \in \N\).

    By uniform convergence, we can choose \(N \in \N\) such that for \(n \geq N\),
    \[
        \abs{f_n(x) - f(x)} < 1, \quad \forall x \in X.
    \]
    Thus, for \(x \in X\) and \(n \geq N\), we can write
    \[
        \abs{f_n(x)} \leq \abs{f_n(x) - f(x)} + \abs{f(x) - f_N(x)} + \abs{f_N(x)} < 2 + M_N.
    \]
    Now set \(M = \max\{M_1, M_2, \dots, M_{N-1}, 2 + M_N\}\). Then for all \(n \in \N\),
    \[
        \abs{f_n(x)} \leq M,
    \]
    which shows that \(\seq{f_n}\) is uniformly bounded.

    \numl{2} Suppose that \(f_n \ra f\), \(g_n \ra g\) uniformly on \(E\), and let \(\epsilon > 0\) be given. By uniform convergence of \(f_n\), \(g_n\), we can choose \(N_1, N_2 \in \N\) such that
    \begin{center}
        \(n \geq N_1 \implies \abs{f_n(x) - f(x)} < \dfrac{\epsilon}{2}\) and \(n \geq N_2 \implies \abs{g_n(x) - g(x)} < \dfrac{\epsilon}{2}\)
    \end{center}
    for all \(x \in E\). Set \(N = \max\{N_1, N_2\}\), we find that for \(n \geq N\),
    \[
        \abs{f_n(x) + g_n(x) - \bigl(f(x) + g(x)\bigr)} \leq \abs{f_n(x) - f(x)} + \abs{g_n(x) - g(x)} < \frac{\epsilon}{2} + \frac{\epsilon}{2} = \epsilon
    \]
    for all \(x \in E\). Thus \(f_n + g_n\) converges uniformly to \(f + g\) on \(E\).

    If \(f_n\), \(g_n\) are bounded, we know that they are both uniformly bounded by the first problem. Additionally, we know that \(f\) is bounded by {\sffamily Theorem 7.15}. Thus there exists \(F, G \in \R \bs \{0\}\) such that \(\abs{f_n(x)} \leq F\), \(\abs{f(x)} \leq F\) and \(\abs{g_n(x)} \leq G\).

    Let \(\epsilon > 0\) be given. Using the uniform convergence of \(f_n\) and \(g_n\), we can choose \(M_1, M_2 \in \N\) such that
    \begin{center}
        \(n \geq M_1 \implies \abs{f_n(x) - f(x)} < \dfrac{\epsilon}{2G}\) and \(n \geq M_2 \implies \abs{g_n(x) - g(x)} < \dfrac{\epsilon}{2F}\)
    \end{center}
    for all \(x \in E\). Set \(M = \max\{M_1, M_2\}\), we find that for \(n \geq M\),
    \[
        \begin{aligned}
            \abs{f_n(x)g_n(x) - f(x)g(x)} & = \abs{f_n(x)g_n(x) - f(x)g_n(x) + f(x)g_n(x) - f(x)g(x)}                                                            \\
                                          & \leq \abs{g_n(x)}\abs{f_n(x) - f(x)} + \abs{f(x)} \abs{g_n(x) - g(x)}                                                \\
                                          & \leq G \cdot \frac{\epsilon}{2G} + F \cdot \frac{\epsilon}{2F} = \frac{\epsilon}{2} + \frac{\epsilon}{2} = \epsilon,
        \end{aligned}
    \]
    for all \(x \in E\). Thus \(f_ng_n\) converges uniformly to \(fg\) on \(E\).

    \numl{3} It is easy to see that \(f_n \ra f \equiv 0\), making \(f\) a continuous function. But the convergence is not uniform. For instance, take \(\epsilon = 1/2\). For all \(n \in \N\), there exists some \(x\in \R\) such that
    \[
        \abs{f_n(x) - f(x)} = \abs{f_n(x)} \geq \frac{1}{2}.
    \]
    \(x_0 = \dfrac{1}{n + 1/2}\) is such \(x\), because
    \[
        \abs{f_n(x_0)} = \sin^2 \dfrac{\pi}{x_0} = \sin^2 \left(n\pi + \frac{\pi}{2}\right) = 1 \geq \frac{1}{2}.
    \]

    Now we calculate \(\sum f_n(x)\). For \(x \leq 0\), \(x \geq 1\), \(\sum f_n(x) = 0\).

    For \(x \in (0, 1)\),
    \begin{enumerate}
        \item If \(x = \dfrac{1}{N}\) for some \(N \in \N\), \(f_n(x) = 0\) for all \(n \in \N\).
        \item Otherwise, there exists \(N \in \N\) such that \(\ds \frac{1}{N + 1} < x < \frac{1}{N}\). Then
              \[
                  f_n(x) = \begin{cases}
                      \sin^2 \dfrac{\pi}{x} & (n = N)    \\
                      0                     & (n \neq N)
                  \end{cases}.
              \]
    \end{enumerate}
    Thus, \(\sum f_n(x) = \sin^2 \dfrac{\pi}{x}\) for \(x \in (0, 1)\). Overall,
    \[
        f(x) = \sum f_n(x) = \begin{cases}
            \sin^2 \dfrac{\pi}{x} & (x \in (0, 1))     \\
            0                     & (\text{otherwise})
        \end{cases}.
    \]

    Since all the terms are non-negative, the series converges absolutely.

    If \(\sum f_n(x)\) were to converge uniformly to \(f\), \(f\) should have been continuous. But since \(f\) is not continuous at \(x = 0\), \(\sum f_n(x)\) cannot converge uniformly.

    \numl{4} The given series can be considered as the sum of the two following series
    \[
        A(x) = x^2 \sum_{n=1}^\infty \frac{(-1)^n}{n^2}, \quad B = \sum_{n=1}^\infty \frac{(-1)^n}{n}
    \]
    because both \(A(x)\) and \(B\) converge.

    Let \(a > 0\) and set \(X = [-a, a]\). It is sufficient to show uniform convergence for \(X\), because \(a\) can be chosen arbitrarily large so that it would contain any bounded interval.
    
    \(A(x)\) converges uniformly on \(X\) by using the Weierstrass \(M\)-test with \(M_n = \dfrac{a^2}{n^2}\). However, the series does not converge absolutely by the comparison test since
    \[
        \abs{(-1)^n \frac{x^2 + n}{n^2}} \geq \frac{n}{n^2} = \frac{1}{n}
    \]
    and the harmonic series diverges.

    \numl{5}
\end{enumerate}
\end{document}
