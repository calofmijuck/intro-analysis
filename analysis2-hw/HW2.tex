%!TEX encoding = utf-8
\documentclass[12pt]{report}
\usepackage{kotex}
\usepackage{amsmath}
\usepackage{amsfonts}
\usepackage{amssymb}
\usepackage{amsthm}
\usepackage{mathtools}
\usepackage{geometry}
\geometry{
    top = 20mm,
    left = 20mm,
    right = 20mm,
    bottom = 20mm
}
\geometry{a4paper}

\pagenumbering{gobble}
\renewcommand{\baselinestretch}{1.4}
\newcommand{\numl}[1]{\item[\large\textbf{\sffamily #1.}]}
\newcommand{\num}[1]{\item[\textbf{\sffamily #1}]}

\newcommand{\ds}{\displaystyle}

\newcommand{\mf}[1]{\mathfrak{#1}}
\newcommand{\mc}[1]{\mathcal{#1}}
\newcommand{\bb}[1]{\mathbb{#1}}
\newcommand{\rmbf}[1]{\mathrm{\mathbf{#1}}}

\newcommand{\inv}{^{-1}}
\newcommand{\adj}{\text{*}}
\newcommand{\bs}{\setminus}
\renewcommand{\subset}{\subseteq}

\newcommand{\norm}[1]{\left\lVert #1 \right\rVert}
\newcommand{\abs}[1]{\left| #1 \right|}
\newcommand{\paren}[1]{\left( #1 \right)}
\newcommand{\seq}[1]{\left\{ #1 \right\}}
\renewcommand{\span}[1]{\left\langle #1 \right\rangle}

\newcommand{\ra}{\rightarrow}
\newcommand{\uc}{\overset{u}{\ra}}
\newcommand{\imp}{\implies}
\newcommand{\mimp}{\(\implies\)}
\newcommand{\mimpd}{\(\impliedby\)}
\newcommand{\miff}{\!\!\(\iff\)}
\newcommand{\mast}{\(\ast\)}

\newcommand{\R}{\mathbb{R}}
\newcommand{\N}{\mathbb{N}}
\newcommand{\Z}{\mathbb{Z}}
\newcommand{\Q}{\mathbb{Q}}
\newcommand{\C}{\mathbb{C}}

\newcommand{\inte}{\mathrm{int}}
\newcommand{\diam}{\text{diam}}
\newcommand{\dist}{\text{dist}}
\newcommand{\lint}[2]{\underline{\int_{#1}^{#2}}}
\newcommand{\uint}[2]{\overline{\int_{#1}^{#2}}}
\renewcommand{\d}[1]{\,d{#1}}

\let\oldexists\exists
\renewcommand{\exists}{\oldexists\,}

\begin{document}
\begin{center}
    \textbf{\Large 해석개론 및 연습 2 과제 \#2}\\
    \large 2017-18570 컴퓨터공학부 이성찬
\end{center}
\begin{enumerate}

    \numl{1} Suppose that \(f\) is not a constant function. Then there exists \(\alpha, \beta \in [0, \infty)\) such that \(f(\alpha) \neq f(\beta)\). Now we show by contradiction that \(\seq{f_n}\) cannot be equicontinuous on \([0, 1]\).

    Set \(\epsilon = \abs{f(\alpha) - f(\beta)}\). There should be a \(\delta > 0\) such that for all \(n \in \N\),
    \begin{center}
        \(x, y \in [0, 1], \abs{x - y} < \delta \implies \abs{f_n(x) - f_n(y)} < \epsilon\).
    \end{center}
    However, we can always choose \(N \in \N\) large enough (\(N > \alpha, \beta\)) so that \(\abs{\alpha - \beta}/N < \delta\).
    For \(x = \alpha/N, y = \beta/N\), \(x, y \in [0, 1]\) and \(\abs{x - y} = \abs{\alpha - \beta}/N < \delta\) but
    \[
        \abs{f_N(x) - f_N(y)} = \abs{f_N\left(\frac{\alpha}{N}\right) - f_N\left(\frac{\alpha}{N}\right)} = \abs{f(a) - f(b)} = \epsilon.
    \]
    Thus such \(\delta > 0\) cannot exist and \(\seq{f_n}\) cannot be equicontinuous, leading to a contradiction. \(f\) has to be a constant function.

    \numl{2} Let \(\epsilon > 0\) be given. Equicontinuity of \(\seq{f_n}\) lets us choose \(\delta > 0\) such that
    \[ \tag{\mast}
        x, y \in K, d(x, y) < \delta \implies \abs{f_n(x) - f_n(y)} < \frac{\epsilon}{3}
    \]
    for all \(n \in \N\). The set \(\ds \bigcup_{x\in K} B_\delta(x)\) is an open cover of \(K\), and there exists a finite subcover because \(K\) is a compact set.
    \begin{center}
        \(\exists x_1, x_2, \dots, x_r \in K\) such that \(\ds K \subset \bigcup_{i=1}^r B_\delta(x_i)\).
    \end{center}

    For all \(x \in K\), there exists \(s \leq r\) such that \(x \in B_\delta(x_s)\). By (\mast),
    \[
        \abs{f_n(x) - f_n(x_s)} < \frac{\epsilon}{3}, \quad (\forall n \in \N)
    \]
    Also since \(\seq{f_n}\) converges pointwise, there exists \(N\in \N\) such that for \(n, m \geq N\),
    \[
        \abs{f_n(x_s) - f_m(x_s)} < \frac{\epsilon}{3}.
    \]
    Therefore, for \(n, m \geq N\),
    \[
        \abs{f_n(x) - f_m(x)} \leq \abs{f_n(x) - f_n(x_s)} + \abs{f_n(x_s) - f_m(x_s)} + \abs{f_m(x_s) - f_m(x)} < \frac{\epsilon}{3} \cdot 3 = \epsilon
    \]
    for all \(x \in K\). \(f_n\) converges uniformly on \(K\).

    \noindent {\sffamily (Counterexample)} Take \(f_n(x) = x/n\) over \(\R\).

    \pagebreak

    \numl{3} We first prove the following claim.

    \quad {\sffamily \bfseries Claim.} \textit{Suppose \(f \geq 0\), \(f\) is continuous on \([a, b]\), and \(\ds\int_a^b f(x)\d{x} = 0\). Then \(f(x) = 0\) on \([a, b]\).}

    \quad {\sffamily Proof.} Suppose there exists some \(x_0 \in [a, b]\) such that \(f(x_0) > 0\). Since \(f\) is continuous, we can find \(\delta > 0\) such that
    \begin{center}
        \(x \in [x_0 - \delta, x_0 + \delta] \implies \abs{f(x) - f(x_0)} < \dfrac{f(x_0)}{2}\).
    \end{center}
    We can conclude that \(f(x) > \dfrac{f(x_0)}{2}\) on \([x_0 - \delta, x_0 + \delta]\). Then,
    \[
        0 = \int_a^b f(x)\d{x} \geq \int_{x_0-\delta}^{x_0+\delta} f(x)\d{x} \geq \int_{x_0-\delta}^{x_0+\delta} \frac{f(x_0)}{2}\d{x} = \delta f(x_0) > 0,
    \]
    which is a contradiction. Therefore \(f(x) = 0\) on \([a, b]\).

    \medskip

    Since \(f\) is continuous on \([0, 1]\) and bounded (Extreme Value Theorem), there exists a sequence of polynomials \(\seq{P_n}\) such that \(P_n \ra f\) uniformly on \([0, 1]\). Let \(P_n(x) = \sum_{k=0}^{\deg P_n} p_{n, k} x^k\). Then
    \[
        \int_0^1 f(x)P_n(x) \d{x} = \int_0^1 f(x)\sum_{k=0}^{\deg P_n} p_{n, k} x^k \d{x} = \sum_{k=0}^{\deg P_n} p_{n, k} \int_0^1 f(x) x^k \d{x} = 0.
    \]
    Note that \(P_n\) is uniformly bounded by its uniform convergence. Since \(P_n\) and \(f\) are both bounded, \(f(x)P_n(x)\) converges uniformly on \([0, 1]\). Therefore,
    \[
        \int_0^1 f(x)^2 \d{x} = \int_0^1 \lim_{n \ra \infty} f(x) P_n(x) \d{x} = \lim_{n\ra\infty} \int_0^1 f(x)P_n(x) \d{x} = 0.
    \]
    (Uniform convergence lets us switch the order of integration and limit.) By the claim above, \(f(x)^2 = 0\) on \([0, 1]\), which implies \(f(x) = 0\) on \([0, 1]\).

    \numl{4} {\sffamily (a)} \(\seq{f_n}\) is uniformly bounded and \(\Q\) is countable. By {\sffamily Theorem 7.23}, there exists a subsequence \(\seq{f_{m_i}}\) of \(\seq{f_n}\) such that \(\seq{f_{m_i}(r)}\) converges for \(\forall r \in \Q\). Now let \(f(x) = \ds \sup_{r \leq x} f(r)\).

    Suppose that \(f\) is continuous at \(x_0 \in \R\). Given \(\epsilon > 0\), there exists \(\delta > 0\) such that
    \[
        \abs{x - x_0} < \delta \implies \abs{f(x) - f(x_0)} < \frac{\epsilon}{2}.
    \]
    Take \(r, s \in \Q\) from the interval \((x_0 - \delta, x_0 + \delta)\) such that \(r \leq x_0 \leq s\). Since \(f_n\) is increasing for all \(n \in \N\), \(f\) should also be increasing. Then
    \[ \tag{\mast}
        f(x_0) -\frac{\epsilon}{2} < f(r) \leq f(x_0) \leq f(s) < f(x_0) + \frac{\epsilon}{2}.
    \]
    Since \(f_{m_i}(r) \ra f(r)\) and \(f_{m_i}(s) \ra f(s)\), set \(i\) large enough so that
    \[
        \abs{f_{m_i}(r) - f(r)} < \frac{\epsilon}{2}, \quad \abs{f_{m_i}(s) - f(s)} < \frac{\epsilon}{2}.
    \]
    Combining these inequalities with (\mast) gives
    \[
        f(x_0) - \epsilon < f(r) - \frac{\epsilon}{2} < f_{m_i}(r), \quad f_{m_i}(s) < f(s) + \frac{\epsilon}{2} < f(x_0) + \epsilon.
    \]
    Since \(f_{m_i}(r) \leq f_{m_i}(x_0) \leq f_{m_i}(s)\), we have that for large \(i\),
    \[
        f(x_0) - \epsilon < f_{m_i}(x_0) < f(x_0) + \epsilon \implies \abs{f_{m_i}(x_0) - f(x_0)} < \epsilon.
    \]
    Thus \(\seq{f_{m_i}}\) converges at points of continuity.

    For a monotone function on \(\R\), the set of discontinuities \(D\) is at most countable. We can apply {\sffamily Theorem 7.23} once again on \(\seq{f_{m_i}}\) to get a subsequence \(\seq{f_{n_k}}\) that converges for all \(x \in D\).

    Now we modify the definition of \(f(x)\) so that for \(x \in D\),
    \[
        f(x) = \lim_{k \ra\infty} f_{n_k}(x).
    \]
    Since \(\seq{f_{n_k}}\) is a subsequence of \(\seq{f_{m_i}}\), \(\seq{f_{n_k}}\) also converges for all \(x \in \R \bs D\), so we can write
    \[
        f(x) = \lim_{k \ra \infty} f_{n_k}(x), \quad x \in \R.
    \]

    {\sffamily (b)} Let \(f\) be a continuous function on a compact set \(K\). Then \(f\) is uniformly continuous.\footnote{Heine-Cantor Theorem.} Let \(\epsilon > 0\) be given.

    Take any \(t \in K\). Since \(f\) is uniformly continuous, we can find \(\delta > 0\) such that
    \[ \tag{\(\star\)}
        x \in (t - \delta, t + \delta) \implies f(t) - \frac{\epsilon}{2} < f(x) < f(t) + \frac{\epsilon}{2}.
    \]
    Let \(g_k = f_{n_k}\). Since \(g_k\) is increasing,
    \[ \tag{\(\spadesuit\)}
        g_k\left(t - \frac{\delta}{2}\right) \leq g_k(t) \leq g_k\left(t + \frac{\delta}{2}\right).
    \]
    Also because \(g_k \ra f\) pointwise, there exists large enough \(N \in \N\) such that
    \[
        k \geq N \implies \abs{g_k\left(t - \frac{\delta}{2}\right) - f\left(t - \frac{\delta}{2}\right)} < \frac{\epsilon}{2} \text{ and } \abs{g_k\left(t + \frac{\delta}{2}\right) - f\left(t + \frac{\delta}{2}\right)} < \frac{\epsilon}{2}.
    \]
    For \(k \geq N\),
    \[
        f\left(t - \frac{\delta}{2}\right) - \frac{\epsilon}{2} < g_k\left(t - \frac{\delta}{2}\right) \text{ and } g_k\left(t + \frac{\delta}{2}\right) < f\left(t + \frac{\delta}{2}\right) + \frac{\epsilon}{2}.
    \]
    From (\(\spadesuit\)),
    \[
        f\left(t - \frac{\delta}{2}\right) - \frac{\epsilon}{2} < g_k(t) < f\left(t +\frac{\delta}{2}\right) + \frac{\epsilon}{2}.
    \]
    Since \(t - \delta / 2, t + \delta/2 \in (t-\delta, t+\delta)\), by \((\star)\),
    \[
        f(t) - \epsilon < f\left(t - \frac{\delta}{2}\right) - \frac{\epsilon}{2} < g_k(t) < f\left(t + \frac{\delta}{2}\right) + \frac{\epsilon}{2} < f(t) + \epsilon.
    \]
    Thus for \(k \geq N\), \(\abs{g_k(t) - f(t)} < \epsilon\) (\(\forall t \in K\)). \(f_{n_k} \ra f\) uniformly on \(K\).

    \pagebreak

    \numl{5} Knowing that \(\norm{f}_2 \geq 0\) for any \(f \in \mc{R}(\alpha)\), we shall prove that for \(f, g \in \mc{R}(\alpha)\),
    \[ \tag{\(\star\)}
        \left(\norm{f}_2 + \norm{g}_2\right)^2 \geq \norm{f+g}_2^2.
    \]
    Let \(I = [a, b]\).
    \[
        \begin{aligned}
            \left(\norm{f}_2 + \norm{g}_2\right)^2 - \norm{f+g}_2^2 & = \norm{f}_2^2 + \norm{g}_2^2 + 2\norm{f}_2\norm{g}_2 + \norm{f+g}_2^2                                                                                                                                      \\
                                                                    & =                                                          \int_I \abs{f}^2 \d{\alpha} + \int_I \abs{g}^2 \d{\alpha} + 2 \left\{\int_I \abs{f}^2 \d{\alpha} \cdot \int_I \abs{g}^2 \d{\alpha}\right\}^{1/2} \\
                                                                    & \qquad - \int_I \abs{f}^2 \d{\alpha} - \int_I \abs{g}^2 \d{\alpha} - 2 \int_I \abs{fg} \d{\alpha}                                                                                                           \\
                                                                    & =                                                         2\left[\left\{\int_I \abs{f}^2 \d{\alpha} \cdot \int_I \abs{g}^2 \d{\alpha}\right\}^{1/2} - \int_I \abs{fg} \d{\alpha}\right]. \quad (\ast)
        \end{aligned}
    \]
    Similarly, the integrals are all positive, so we instead prove the following.

    \medskip

    \quad {\sffamily \bfseries Claim.} \(\ds \int_I \abs{f}^2 \d{\alpha} \cdot \int_I \abs{g}^2 \d{\alpha} - \left(\int_I \abs{fg} \d{\alpha}\right)^2 \geq 0\).

    \quad {\sffamily Proof.} Consider the function \((t\abs{f} - \abs{g})^2\), where \(t\) is some real constant. Integrating this function with respect to \(\alpha\) should be non-negative. Therefore,
    \[
        \int_I \bigl(t\abs{f} - \abs{g}\bigr)^2 \d{\alpha} = t^2 \int_I \abs{f}^2 \d{\alpha} - 2t \int_I \abs{fg}\d{\alpha} + \int_I \abs{g}^2\d{\alpha} \geq 0.
    \]
    Treating the above expression as a quadratic of \(t\), the discriminant should be non-positive.
    \[
        D/4 = \left(\int_I \abs{fg} \d{\alpha}\right)^2 - \int_I \abs{f}^2 \d{\alpha} \cdot \int_I \abs{g}^2 \d{\alpha} \leq 0,
    \]
    giving the desired result. Equality holds when \(f, g\) are linearly dependent.

    \medskip

    Hence (\mast) \(\geq 0\), proving \((\star)\). For \(f, g, h \in \mc{R}(\alpha)\), replacing \(f\) by \(f - g\), \(g\) by \(g - h\) gives
    \[
        \norm{f - h}_2 \leq \norm{f-g}_2 + \norm{g-h}_2,
    \]
    proving the original inequality.

    \pagebreak

    \numl{6} For any given partition \(P = \{a = x_0, x_1, \dots, x_n = b\}\) of \([a, b]\), define
    \[
        g(t) = \frac{x_i - t}{x_i - x_{i-1}} f(x_{i-1}) + \frac{t - x_{i-1}}{x_i - x_{i-1}} f(x_i)
    \]
    for \(t \in [x_{i-1}, x_i]\). \(g(t)\) is piecewise continuous (linear) and for points in the partition except for the both ends, the left/right limit of \(g(t)\) at \(t = x_i\) are the same. Thus \(g(t)\) is continuous on \([a, b]\).

    Since \(f \in \mc{R}(\alpha)\), \(f\) is bounded on \([a, b]\). Let \(\abs{f} < M\) for some positive real \(M\).

    For any interval \(I_i = [x_{i-1}, x_i] \subset [a, b]\), \(\ds \inf_{x \in I_i} f(x) \leq f(x) \leq \sup_{x \in I_i} f(x)\) by definition. Also, since \(g(x)\) is linear,
    \[
        \inf_{x \in I_i} f(x) \leq \min\{f(x_{i-1}), f(x_i)\} \leq g(x) \leq \max\{f(x_{i-1}), f(x_i)\} \leq \sup_{x \in I_i} f(x).
    \] Therefore,
    \[
        0 \leq \abs{f(x) - g(x)} \leq \sup_{x\in I_i} f(x) - \inf_{x\in I_i}f(x) \implies \abs{f(x) - g(x)}^2 \leq \bigl(\sup_{x\in I_i} f(x) - \inf_{x\in I_i}f(x)\bigr)^2.
    \]
    Additionally,
    \[
        \sup_{x\in I_i} f(x) - \inf_{x\in I_i}f(x) < 2M.
    \]

    Now let \(\epsilon > 0\) be given. Since \(f\) is integrable, we can choose a partition \(P\) of \([a, b]\) such that
    \[
        U(P, f, \alpha) - L(P, f, \alpha) < \frac{\epsilon^2}{2M}.
    \]
    Then we can define a continuous function \(g(x)\) as above, and
    \[
        \begin{aligned}
            \norm{f-g}_2^2 & \leq U(P, \abs{f-g}^2, \alpha)                                                                                                                   \\
                           & = \sum_{i=1}^n \sup_{x \in [x_{i-1}, x_i]} \abs{f(x) - g(x)}^2 \bigl(\alpha(x_i) - \alpha(x_{i-1})\bigr)                                         \\
                           & \leq \sum_{i=1}^n \left(\sup_{x \in [x_{i-1}, x_i]} f(x) - \inf_{x \in [x_{i-1}, x_i]} f(x) \right)^2 \bigl(\alpha(x_i) - \alpha(x_{i-1})\bigr)  \\
                           & \leq 2M \sum_{i=1}^n \left(\sup_{x \in [x_{i-1}, x_i]} f(x) - \inf_{x \in [x_{i-1}, x_i]} f(x) \right) \bigl(\alpha(x_i) - \alpha(x_{i-1})\bigr) \\
                           & = 2M\big(U(P, f, \alpha) - L(P, f, \alpha)\big) < 2M \cdot \frac{\epsilon^2}{2M} = \epsilon^2.
        \end{aligned}
    \]
    Therefore \(\norm{f-g}_2 < \epsilon\), as desired.
\end{enumerate}
\end{document}
