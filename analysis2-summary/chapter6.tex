\chapter{Sequence of Functions}

\section{Sequence of Continuous Functions}

\defn. (Sequence of Functions) Let \(X \subseteq \R^n\), \(Y \subseteq \R^m\). Given
\[
	f_n: X\ra Y
\]
for each \(n\in \N\), we call \(\span{f_n}\) a \textbf{sequence of functions from \(X\) to \(Y\)}.

\defn. (Pointwise Convergence) The sequence \(\span{f_n}\) \textbf{converges pointwise} to the function \(f: X \ra Y\) if and only if
\[
	\lim_{n\ra \infty} f_n(x) = f(x)
\]
for each \(x \in X\). In other words, given \(\epsilon > 0\) and for all \(x \in X\),
\begin{center}
	\(\exists N\in \N\)\quad  s.t.\quad \(n \geq N \implies \norm{f_n(x) - f(x)} < \epsilon\).\footnote{여기서 주의해야 할 점은 자연수 \(N\) 이 양수 \(\epsilon > 0\) 뿐 아니라 정의역의 점 \(x\in X\) 에도 의존한다는 점이다.}
\end{center}

\defn. (Sequence of Continuous Functions) \(\span{f_n}\) is a sequence of continuous functions if and only if \(f_n\) is continuous for all \(n\in \N\).

\textbf{Question}. \textit{Suppose \(\span{f_n}\) is a sequence of continuous functions that converges pointwise to \(f\). Is \(f\) also continuous?}

\defn. (Uniform Convergence) Let \(\span{f_n}\) be a sequence of functions defined on \(X \subseteq \R^n\) and let \(f\) be a function defined on \(X\). We say that \(\span{f_n}\) is \textbf{uniformly convergent on \(X\)} if and only if for any given \(\epsilon > 0\), there exists \(N\in \N\) such that
\[
	n\geq N,\; x\in X \implies \norm{f_n(x) - f(x)} < \epsilon
\]

\clearpage
\prob{6.1.1.} Following are equivalent.
\begin{enumerate}
	\item \(\span{f_n}\) is uniformly convergent on \(X\).
	\item \(\ds \lim_{n\ra\infty} \norm{f_n-f}_{\sup} := \lim_{n\ra \infty} \sup \left\{\norm{f_n-f}: x\in X\right\} = 0\).
\end{enumerate}

\pf. (1\mimp2) Uniformly convergent on \(X\) \mimp \(\forall \epsilon > 0\), \(\exists N \in \N\) s.t. \(n \geq N, x\in X \implies \norm{f_n(x) - f(x)} < \epsilon/2\). Then \(0 \leq \sup\left\{\norm{f_n(x) - f(x)} : x \in X \right\} < \epsilon/2 < \epsilon\), and we have the desired result.
(2\mimp1) If \(\ds \lim_{n\ra \infty} \sup \left\{\norm{f_n-f}: x\in X\right\} = 0\), for any \(\epsilon > 0\), there exists \(N\in \N\) such that \(n\geq N,\; \sup\left\{\norm{f_n(x) - f(x)} : x \in X \right\} < \epsilon / 2\). Then \(\norm{f_n(x) - f(x)}\) should be less than \(\epsilon\) for all \(x\in X\), and thus \(\span{f_n}\) is uniformly convergent.

\prob{6.1.2.} \(\ds f_n(x) = \frac{1}{n}x\) is not uniformly convergent on \(\R\).

\pf. Suppose \(\span{f_n}\) is converges uniformly on \(\R\) to 0. Then for any given \(\epsilon > 0\), there exists \(N\in\N\) such that for \(n\geq N, x\in \R \implies \abs{\frac{1}{n}x} < \epsilon\). But this can't be true, because for any \(\epsilon\), we can take \(x\) to be as large as we want. Take \(x = 2\epsilon n\) for example, then \(\abs{\frac{1}{n}x} = 2\epsilon > \epsilon\). Contradiction.

\thm{6.1.1.} If a sequence \(\span{f_n}\) of continuous functions from \(X\) to \(Y\) converges uniformly to \(f: X \ra Y\), then \(f\) is a continuous function.

\pf. Given \(\epsilon > 0\) and \(x_0 \in X\), choose large enough \(N\in\N\) such that
\[
	x \in X \implies \norm{f(x) - f_N(x)} < \frac{\epsilon}{3}
\]
Since \(f_N\) is continuous, there exists \(\delta > 0\) such that
\[
	x\in X,\; \norm{x - x_0} < \delta \implies \norm{f_N(x) - f_N(x_0)} < \frac{\epsilon}{3}
\]
If \(x\in X\) and \(\norm{x - x_0} < \delta\), then we have
\[
	\begin{aligned}
		\norm{f(x) - f(x_0)} & \leq \norm{f(x) - f_N(x)} + \norm{f_N(x) - f_N(x_0)} + \norm{f_N(x_0) -f(x_0)} \\
		                     & = \frac{\epsilon}{3} + \frac{\epsilon}{3} + \frac{\epsilon}{3} = \epsilon
	\end{aligned}
\]
So we can conclude that \(f\) is continuous at \(x_0\). (Also note that uniform convergence implies pointwise convergence.)

\prop{.} If \(\span{f_n}\) converges uniformly to \(f: X\ra Y\) and if \(\ds \lim_{x \ra x_0} f_n(x)\) exists for all \(n\), the following holds.
\[
	\lim_{n\ra \infty} \lim_{x \ra x_0} f_n(x) = \lim_{x \ra x_0} \lim_{n \ra \infty} f_n(x), \qquad x_0 \in X'
\]
\pf. Let \(\ds \lim_{x \ra x_0} f_n(x) = a_n\) for \(n\in \N\). We want to show that \(\ds \lim_{n\ra\infty} a_n = \lim_{x\ra x_0} f(x)\).
\newline
First, we show that \(\{a_n\}\) converges by proving that \(\{a_n\}\) is a Cauchy sequence.
\newline
Take any \(\epsilon > 0\). By uniform convergence, there exists \(N\in \N\) such that
\[
	n\geq N, \; x \in X \implies \norm{f_n(x) - f(x)} < \epsilon
\]
Furthermore, because \(\ds \lim_{x\ra x_0} f_n(x) = a_n\), there exists \(\delta > 0\) such that
\[
	\norm{x - x_0} < \delta \implies \norm{f_n(x) - a_n} < \epsilon
\]
Take \(m, n \geq N\). Then we have
\[
	\begin{aligned}
		\norm{a_n - a_m} \leq \norm{a_n - f_n(x)} + \norm{f_n(x) - f(x)} + \norm{f(x) - f_m(x)} + \norm{f_m(x)-a_m} < 4\epsilon
	\end{aligned}
\]
, since we can take \(x\) to be as close as we want to \(x_0\). Therefore \(\{a_n\}\) is a Cauchy sequence, let its limit be \(a\).
\newline
Now it is enough to show that \(\ds \lim_{x\ra x_0} f(x) = a\). For \(\epsilon > 0\), there exists \(\delta > 0\) such that
\[
	\begin{aligned}
		\norm{x - x_0} < \delta \implies
		\norm{f(x) - a} & \leq \norm{f(x) - f_n(x)} + \norm{f_n(x) - a_n} + \norm{a_n - a}           \\
		                & < \frac{\epsilon}{3} + \frac{\epsilon}{3} + \frac{\epsilon}{3} = \epsilon.
	\end{aligned}
\]
The second inequality holds because each of the terms are smaller than \(\epsilon / 3\) due to uniform convergence, \(\ds \lim_{x \ra x_0} f_n(x) = a_n\), \(\ds \lim_{n \ra \infty} a_n = a\), respectively.\footnote{조건 \(x \in X'\) 은 어디서 이용된 것일까?}

\thm{6.1.2.} The following are equivalent for sequence of functions \(\span{f_n}\) from \(X\) to \(Y\).
\begin{enumerate}
	\item \(\span{f_n}\) converges uniformly to \(f\).
	\item For all \(\epsilon > 0\), there exists \(N \in \N\) such that
	      \begin{equation}\tag{\textasteriskcentered}
			m, n \geq N, \; x\in X \implies \norm{f_m(x) - f_n(x)} < \epsilon 
		  \end{equation}
\end{enumerate}

\pf. (1\mimp2) (Similar to the proof of Proposition 2.3.6).
\newline
(2\mimp1) Fix \(x\in X\). Then we directly see that \(\{f_n(x)\}\) is a Cauchy sequence. Suppose its limit is \(f(x)\). For \(\forall \epsilon > 0\), take \(N\) such that (\textasteriskcentered) holds and set \(m\ra \infty\). Then for each \(n \geq N\), \(\norm{f_n-f}_{\sup} \leq \epsilon\), thus \(\span{f_n}\) converges uniformly to \(f\).
